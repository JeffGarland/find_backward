%!TEX root = std.tex
\rSec0[std2.ranges]{Ranges library}

\rSec1[std2.ranges.general]{General}

\pnum
This \changed{C}{Subc}lause describes components for dealing with ranges of elements.

\pnum
The following subclauses describe
range and view requirements, and
components for
range primitives
as summarized in Table~\ref{tab:ranges.lib.summary}.

\begin{libsumtab}{Ranges library summary}{tab:ranges.lib.summary}
  \ref{std2.range.access}        & Range access      & \tcode{<\changed{experimental/ranges}{std2}/range>} \\
  \ref{std2.range.primitives}    & Range primitives  & \\ \rowsep
  \ref{std2.ranges.requirements} & Requirements      & \\
\end{libsumtab}

\rSec1[std2.ranges.decaycopy]{decay_copy}

\ednote{TODO: Replace the definition of [thread.decaycopy] with this definition, and
move it where both standard libraries can make use of it. Location TBD.}

\pnum
Several places in this \changed{C}{Subc}lause use the expression \tcode{\textit{DECAY_COPY}(x)},
which is expression-equivalent to:
\begin{codeblock}
  decay_t<decltype((x))>(x)
\end{codeblock}

\rSec1[std2.range.synopsis]{Header \tcode{<std2/range>} synopsis}

\indexlibrary{\idxhdr{std2/range}}%
\begin{codeblock}
#include <@\changed{experimental/ranges}{std2}@/iterator>

namespace @\changed{std \{ namespace experimental \{ namespace ranges}{std2}@ { inline namespace v1 {
  // \ref{std2.range.access}, range access:
  @\removed{namespace \{}@
    @\added{inline}@ constexpr @\unspec@ begin = @\unspec@;
    @\added{inline}@ constexpr @\unspec@ end = @\unspec@;
    @\added{inline}@ constexpr @\unspec@ cbegin = @\unspec@;
    @\added{inline}@ constexpr @\unspec@ cend = @\unspec@;
    @\added{inline}@ constexpr @\unspec@ rbegin = @\unspec@;
    @\added{inline}@ constexpr @\unspec@ rend = @\unspec@;
    @\added{inline}@ constexpr @\unspec@ crbegin = @\unspec@;
    @\added{inline}@ constexpr @\unspec@ crend = @\unspec@;
  @\removed{\}}@

  // \ref{std2.range.primitives}, range primitives:
  @\removed{namespace \{}@
    @\added{inline}@ constexpr @\unspec@ size = @\unspec@;
    @\added{inline}@ constexpr @\unspec@ empty = @\unspec@;
    @\added{inline}@ constexpr @\unspec@ data = @\unspec@;
    @\added{inline}@ constexpr @\unspec@ cdata = @\unspec@;
  @\removed{\}}@

  template <class T>
  using iterator_t = decltype(@\changed{ranges}{::std2}@::begin(declval<T&>()));

  template <class T>
  using sentinel_t = decltype(@\changed{ranges}{::std2}@::end(declval<T&>()));

  template <class>
  constexpr bool disable_sized_range = false;

  template <class T>
  struct enable_view { };

  struct view_base { };

  // \ref{std2.ranges.requirements}, range requirements:

  // \ref{std2.ranges.range}, Range:
  template <class T>
  concept @\removed{bool}@ Range = @\seebelow@;

  // \ref{std2.ranges.sized}, SizedRange:
  template <class T>
  concept @\removed{bool}@ SizedRange = @\seebelow@;

  // \ref{std2.ranges.view}, View:
  template <class T>
  concept @\removed{bool}@ View = @\seebelow@;

  // \ref{std2.ranges.common}, @\changed{BoundedRange}{CommonRange}@:
  template <class T>
  concept @\removed{bool}@ @\changed{BoundedRange}{CommonRange}@ = @\seebelow@;

  // \ref{std2.ranges.input}, InputRange:
  template <class T>
  concept @\removed{bool}@ InputRange = @\seebelow@;

  // \ref{std2.ranges.output}, OutputRange:
  template <class R, class T>
  concept @\removed{bool}@ OutputRange = @\seebelow@;

  // \ref{std2.ranges.forward}, ForwardRange:
  template <class T>
  concept @\removed{bool}@ ForwardRange = @\seebelow@;

  // \ref{std2.ranges.bidirectional}, BidirectionalRange:
  template <class T>
  concept @\removed{bool}@ BidirectionalRange = @\seebelow@;

  // \ref{std2.ranges.random.access}, RandomAccessRange:
  template <class T>
  concept @\removed{bool}@ RandomAccessRange = @\seebelow@;
}}@\removed{\}\}}@
\end{codeblock}

\rSec1[std2.range.access]{Range access}

\pnum
In addition to being available via inclusion of the \tcode{<\changed{experimental/ranges}{std2}/range>}
header, the customization point objects in \ref{std2.range.access} are
available when \tcode{<\changed{experimental/ranges}{std2}/iterator>} is included.

\ednote{The customization point objects in this subsection all have deprecated
behavior that permits them to work with rvalues. This is for compatability with
the similarly named facilities in namespace \tcode{std}. The authors intend to
replace the deprecated behavior with proper support for rvalue ranges, pending
some ongoing work on an improved design. We hope to bring forward such a design
in the summer meeting in Geneva later this year. See discussion in issue
\href{https://github.com/ericniebler/stl2/issues/547}{stl2\#547}.}

\rSec2[std2.range.access.begin]{\tcode{begin}}
\pnum
The name \tcode{begin} denotes a customization point
 object~(\cxxref{customization.point.object}). The expression
\tcode{\changed{ranges}{::std2}::begin(E)} for some subexpression \tcode{E} is expression-equivalent to:

\begin{itemize}
\item
  \tcode{\changed{ranges}{::std2}::begin(static_cast<const T\&>(E))} if \tcode{E} is an rvalue of
  type \tcode{T}. This usage is deprecated.
  \enternote This deprecated usage exists so that
  \tcode{\changed{ranges}{::std2}::begin(E)} behaves similarly to \tcode{std::begin(E)}
  as defined in ISO/IEC 14882 when \tcode{E} is an rvalue. \exitnote

\item
  Otherwise, \tcode{(E) + 0} if \tcode{E} has array
  type~(\cxxref{basic.compound}).

\item
  Otherwise, \tcode{\textit{DECAY_COPY}((E).begin())} if it is a valid expression and its type \tcode{I} meets the
  syntactic requirements of \tcode{Iterator<I>}. If
  \tcode{Iterator} is not satisfied, the program is ill-formed
  with no diagnostic required.

\item
  Otherwise, \tcode{\textit{DECAY_COPY}(begin(E))} if it is a valid expression and its type \tcode{I} meets the
  syntactic requirements of \tcode{Iterator<I>} with overload
  resolution performed in a context that includes the declaration
  \tcode{\added{template <class T>} void begin(\changed{auto}{T}\&) = delete;} and does not include
  a declaration of \tcode{\changed{ranges}{::std2}::begin}. If \tcode{Iterator}
  is not satisfied, the program is ill-formed with no diagnostic
  required.

\item
  Otherwise, \tcode{\changed{ranges}{::std2}::begin(E)} is ill-formed.
\end{itemize}

\pnum
\enternote Whenever \tcode{\changed{ranges}{::std2}::begin(E)} is a valid expression, its
type satisfies \tcode{Iterator}. \exitnote

\rSec2[std2.range.access.end]{\tcode{end}}
\pnum
The name \tcode{end} denotes a customization point
object~(\cxxref{customization.point.object}). The expression
\tcode{\changed{ranges}{::std2}::end(E)} for some subexpression \tcode{E} is expression-equivalent to:

\begin{itemize}
\item
  \tcode{\changed{ranges}{::std2}::end(static_cast<const T\&>(E))} if \tcode{E} is an rvalue of
  type \tcode{T}. This usage is deprecated.
  \enternote This deprecated usage exists so that
  \tcode{\changed{ranges}{::std2}::end(E)} behaves similarly to \tcode{std::end(E)}
  as defined in ISO/IEC 14882 when \tcode{E} is an rvalue. \exitnote

\item
  Otherwise, \tcode{(E) + extent\added{_v}<T>\removed{::value}} if \tcode{E} has array
  type~(\cxxref{basic.compound}) \tcode{T}.

\item
  Otherwise, \tcode{\textit{DECAY_COPY}((E).end())} if it is a valid expression and its type \tcode{S} meets the
  syntactic requirements of
  \tcode{Sentinel<\brk{}S, decltype(\brk{}\changed{ranges}{::std2}::\brk{}begin(E))>}. If
  \tcode{Sentinel} is not satisfied, the program is ill-formed with
  no diagnostic required.

\item
  Otherwise, \tcode{\textit{DECAY_COPY}(end(E))} if it is a valid expression and its type \tcode{S} meets the
  syntactic requirements of
  \tcode{Sentinel<\brk{}S, decltype(\brk{}\changed{ranges}{::std2}::\brk{}begin(E))>} with overload
  resolution performed in a context that includes the declaration
  \tcode{\added{template <class T>} void end(\changed{auto}{T}\&) = delete;} and does not include
  a declaration of \tcode{\changed{ranges}{::std2}::end}. If \tcode{Sentinel} is not
  satisfied, the program is ill-formed with no diagnostic required.

\item
  Otherwise, \tcode{\changed{ranges}{::std2}::end(E)} is ill-formed.
\end{itemize}

\pnum
\enternote Whenever \tcode{\changed{ranges}{::std2}::end(E)} is a valid expression, the
types of \tcode{\changed{ranges}{::std2}::end(E)} and \tcode{\changed{ranges}{::std2}::\brk{}begin(E)} satisfy
\tcode{Sentinel}. \exitnote

\rSec2[std2.range.access.cbegin]{\tcode{cbegin}}
\pnum
The name \tcode{cbegin} denotes a customization point
object~(\cxxref{customization.point.object}). The expression
\tcode{\changed{ranges}{::std2}::\brk{}cbegin(E)} for some subexpression \tcode{E} of type \tcode{T}
is expression-equivalent to \tcode{\changed{ranges}{::std2}::\brk{}begin(static_cast<const T\&>(E))}.

\pnum
Use of \tcode{\changed{ranges}{::std2}::cbegin(E)} with rvalue \tcode{E} is deprecated.
\enternote This deprecated usage exists so that \tcode{\changed{ranges}{::std2}::cbegin(E)}
behaves similarly to \tcode{std::cbegin(E)} as defined in ISO/IEC 14882 when
\tcode{E} is an rvalue. \exitnote

\pnum
\enternote Whenever \tcode{\changed{ranges}{::std2}::cbegin(E)} is a valid expression, its
type satisfies \tcode{Iterator}. \exitnote

\rSec2[std2.range.access.cend]{\tcode{cend}}
\pnum
The name \tcode{cend} denotes a customization point
object~(\cxxref{customization.point.object}). The expression
\tcode{\changed{ranges}{::std2}::cend(E)} for some subexpression \tcode{E} of type \tcode{T}
is expression-equivalent to \tcode{\changed{ranges}{::std2}::end(static_cast<const T\&>(E))}.

\pnum
Use of \tcode{\changed{ranges}{::std2}::cend(E)} with rvalue \tcode{E} is deprecated.
\enternote This deprecated usage exists so that \tcode{\changed{ranges}{::std2}::\brk{}cend(E)}
behaves similarly to \tcode{std::cend(E)} as defined in ISO/IEC 14882 when
\tcode{E} is an rvalue. \exitnote

\pnum
\enternote Whenever \tcode{\changed{ranges}{::std2}::cend(E)} is a valid expression, the
types of \tcode{\changed{ranges}{::std2}::cend(E)} and \tcode{\changed{ranges}{::std2}::\brk{}cbegin(E)} satisfy
\tcode{Sentinel}. \exitnote

\rSec2[std2.range.access.rbegin]{\tcode{rbegin}}
\pnum
The name \tcode{rbegin} denotes a customization point
object~(\cxxref{customization.point.object}). The expression
\tcode{\changed{ranges}{::std2}::rbegin(E)} for some subexpression \tcode{E} is expression-equivalent
to:

\begin{itemize}
\item
  \tcode{\changed{ranges}{::std2}::rbegin(static_cast<const T\&>(E))} if \tcode{E} is an rvalue of
  type \tcode{T}. This usage is deprecated.
  \enternote This deprecated usage exists so that
  \tcode{\changed{ranges}{::std2}::rbegin(E)} behaves similarly to \tcode{std::rbegin(E)}
  as defined in ISO/IEC 14882 when \tcode{E} is an rvalue. \exitnote

\item
  Otherwise, \tcode{\textit{DECAY_COPY}((E).rbegin())} if it is a valid expression and its type \tcode{I} meets the
  syntactic requirements of \tcode{Iterator<I>}. If \tcode{Iterator}
  is not satisfied, the program is ill-formed with no diagnostic
  required.

\item
  Otherwise, \tcode{make_reverse_iterator(\changed{ranges}{::std2}::end(E))} if both
  \tcode{\changed{ranges}{::std2}::begin(E)} and \tcode{\changed{ranges}{::std2}::\brk{}end(\brk{}E)} are valid expressions of the same
  type \tcode{I} which meets the syntactic requirements of
  \tcode{Bi\-direct\-ional\-Iterator<I>}~(\ref{std2.iterators.bidirectional}).

\item
  Otherwise, \tcode{\changed{ranges}{::std2}::rbegin(E)} is ill-formed.
\end{itemize}

\pnum
\enternote Whenever \tcode{\changed{ranges}{::std2}::rbegin(E)} is a valid expression, its
type satisfies \tcode{Iterator}. \exitnote

\rSec2[std2.range.access.rend]{\tcode{rend}}
\pnum
The name \tcode{rend} denotes a customization point
object~(\cxxref{customization.point.object}). The expression
\tcode{\changed{ranges}{::std2}::rend(E)} for some subexpression \tcode{E} is expression-equivalent to:

\begin{itemize}
\item
  \tcode{\changed{ranges}{::std2}::rend(static_cast<const T\&>(E))} if \tcode{E} is an rvalue of
  type \tcode{T}. This usage is deprecated.
  \enternote This deprecated usage exists so that
  \tcode{\changed{ranges}{::std2}::rend(E)} behaves similarly to \tcode{std::rend(E)}
  as defined in ISO/IEC 14882 when \tcode{E} is an rvalue. \exitnote

\item
  Otherwise, \tcode{\textit{DECAY_COPY}((E).rend())} if it is a valid expression and its type \tcode{S} meets the
  syntactic requirements of
  \tcode{Sentinel<\brk{}S, decltype(\brk{}\changed{ranges}{::std2}::\brk{}rbegin(E))>}. If
  \tcode{Sentinel} is not satisfied, the program is ill-formed with
  no diagnostic required.

\item
  Otherwise, \tcode{make_reverse_iterator(\changed{ranges}{::std2}\colcol{}begin(E))} if both
  \tcode{\changed{ranges}{::std2}::begin(E)} and \tcode{\changed{ranges}{::std2}\colcol{}end(\brk{}E)} are valid expressions of the same
  type \tcode{I} which meets the syntactic requirements of
  \tcode{Bi\-dir\-ect\-ion\-al\-It\-er\-at\-or<I>}~(\ref{std2.iterators.bidirectional}).

\item
  Otherwise, \tcode{\changed{ranges}{::std2}::rend(E)} is ill-formed.
\end{itemize}

\pnum
\enternote Whenever \tcode{\changed{ranges}{::std2}::rend(E)} is a valid expression, the
types of \tcode{\changed{ranges}{::std2}::\brk{}rend(E)} and \tcode{\changed{ranges}{::std2}::\brk{}rbegin(E)} satisfy
\tcode{Sentinel}. \exitnote

\rSec2[std2.range.access.crbegin]{\tcode{crbegin}}
\pnum
The name \tcode{crbegin} denotes a customization point
object~(\cxxref{customization.point.object}). The expression
\tcode{\changed{ranges}{::std2}::\brk{}crbegin(E)} for some subexpression \tcode{E} of type \tcode{T}
is expression-equivalent to \tcode{\changed{ranges}{::std2}::\brk{}rbegin(static_cast<const T\&>(E))}.

\pnum
Use of \tcode{\changed{ranges}{::std2}::crbegin(E)} with rvalue \tcode{E} is deprecated.
\enternote This deprecated usage exists so that \tcode{\changed{ranges}{::std2}::crbegin(E)}
behaves similarly to \tcode{std::crbegin(E)} as defined in ISO/IEC 14882 when
\tcode{E} is an rvalue. \exitnote

\pnum
\enternote Whenever \tcode{\changed{ranges}{::std2}::crbegin(E)} is a valid expression, its
type satisfies \tcode{Iterator}. \exitnote

\rSec2[std2.range.access.crend]{\tcode{crend}}
\pnum
The name \tcode{crend} denotes a customization point
object~(\cxxref{customization.point.object}). The expression
\tcode{\changed{ranges}{::std2}::crend(E)} for some subexpression \tcode{E} of type \tcode{T}
is expression-equivalent to \tcode{\changed{ranges}{::std2}::rend(static_cast<const T\&>(E))}.

\pnum
Use of \tcode{\changed{ranges}{::std2}::crend(E)} with rvalue \tcode{E} is deprecated.
\enternote This deprecated usage exists so that \tcode{\changed{ranges}{::std2}::crend(E)}
behaves similarly to \tcode{std::crend(E)} as defined in ISO/IEC 14882 when
\tcode{E} is an rvalue. \exitnote

\pnum
\enternote Whenever \tcode{\changed{ranges}{::std2}::crend(E)} is a valid expression, the
types of \tcode{\changed{ranges}{::std2}::crend(E)} and \tcode{\changed{ranges}{::std2}::\brk{}crbegin(\brk{}E)} satisfy
\tcode{Sentinel}. \exitnote

\rSec1[std2.range.primitives]{Range primitives}

\pnum
In addition to being available via inclusion of the \tcode{<\changed{experimental/ranges}{std2}/range>}
header, the customization point objects in \ref{std2.range.primitives} are
available when \tcode{<\changed{experimental/ranges}{std2}/iterator>} is included.

\rSec2[std2.range.primitives.size]{\tcode{size}}
\pnum
The name \tcode{size} denotes a customization point
object~(\cxxref{customization.point.object}). The expression
\tcode{\changed{ranges}{::std2}::size(E)} for some subexpression \tcode{E} with type
\tcode{T} is expression-equivalent to:

\begin{itemize}
\item
  \tcode{\textit{DECAY_COPY}(extent\added{_v}<T>\removed{::value})} if \tcode{T} is an array
  type~(\cxxref{basic.compound}).

\item
  Otherwise, \tcode{\textit{DECAY_COPY}(static_cast<const T\&>(E).size())} if it is a valid expression and its type \tcode{I}
  satisfies \tcode{Integral<I>} and
  \tcode{disable_\-sized_\-range<T>}~(\ref{std2.ranges.sized}) is
  \tcode{false}.

\item
  Otherwise, \tcode{\textit{DECAY_COPY}(size(static_cast<const T\&>(E)))} if it is a valid expression and its type \tcode{I}
  satisfies \tcode{Integral<I>} with overload resolution
  performed in a context that includes the declaration
  \tcode{\added{template <class T>} void size(const \changed{auto}{T}\&) = delete;} and does not include
  a declaration of \tcode{\changed{ranges}{::std2}::size}, and
  \tcode{disable_\-sized_\-range<T>} is \tcode{false}.

\item
  Otherwise,
  \tcode{\textit{DECAY_COPY}(\changed{ranges}{::std2}::cend(E) - \changed{ranges}{::std2}::cbegin(E))}, except that \tcode{E}
  is only evaluated once, if it is a valid expression and the types \tcode{I} and \tcode{S} of
  \tcode{\changed{ranges}{::std2}::cbegin(E)} and \tcode{\changed{ranges}{::std2}\colcol{}cend(\brk{}E)} meet the
  syntactic requirements of
  \tcode{SizedSentinel<S, I>}~(\ref{std2.iterators.sizedsentinel}) and
  \tcode{Forward\-Iter\-at\-or<I>}. If \tcode{SizedSentinel} and
  \tcode{Forward\-Iter\-at\-or} are not satisfied, the program is ill-formed with no
  diagnostic required.

\item
  Otherwise, \tcode{\changed{ranges}{::std2}::size(E)} is ill-formed.
\end{itemize}

\pnum
\enternote Whenever \tcode{\changed{ranges}{::std2}::size(E)} is a valid expression, its
type satisfies \tcode{Integral}. \exitnote

\rSec2[std2.range.primitives.empty]{\tcode{empty}}
\pnum
The name \tcode{empty} denotes a customization point
object~(\cxxref{customization.point.object}). The expression
\tcode{\changed{ranges}{::std2}::empty(E)} for some subexpression \tcode{E} is
expression-equivalent to:

\begin{itemize}
\item
  \tcode{bool((E).empty())} if it is a valid expression.

\item
  Otherwise, \tcode{\changed{ranges}{::std2}::size(E) == 0} if it is a valid expression.

\item
  Otherwise, \tcode{bool(\changed{ranges}{::std2}::begin(E) == \changed{ranges}{::std2}::end(E))},
  except that \tcode{E} is only evaluated once, if it is a valid expression and the type of
  \tcode{\changed{ranges}{::std2}::begin(E)} satisfies \tcode{ForwardIterator}.

\item
  Otherwise, \tcode{\changed{ranges}{::std2}::empty(E)} is ill-formed.
\end{itemize}

\pnum
\enternote Whenever \tcode{\changed{ranges}{::std2}::empty(E)} is a valid expression, it
has type \tcode{bool}. \exitnote

\rSec2[std2.range.primitives.data]{\tcode{data}}
\pnum
The name \tcode{data} denotes a customization point
object~(\cxxref{customization.point.object}). The expression
\tcode{\changed{ranges}{::std2}::data(E)} for some subexpression \tcode{E} is
expression-equivalent to:

\begin{itemize}
\item
  \tcode{\changed{ranges}{::std2}::data(static_cast<const T\&>(E))} if \tcode{E} is an rvalue of
  type \tcode{T}. This usage is deprecated. \enternote
  This deprecated usage exists so that \tcode{\changed{ranges}{::std2}::data(E)} behaves
  similarly to \tcode{std::data(E)} as defined in the \Cpp Working
  Paper when \tcode{E} is an rvalue. \exitnote

\item
  Otherwise, \tcode{\textit{DECAY_COPY}((E).data())} if it is a valid expression of pointer to object type.

\item
  Otherwise, \tcode{\changed{ranges}{::std2}::begin(E)} if it is a valid expression of pointer to object type.

\item
  Otherwise, \tcode{\changed{ranges}{::std2}::data(E)} is ill-formed.
\end{itemize}

\pnum
\enternote Whenever \tcode{\changed{ranges}{::std2}::data(E)} is a valid expression, it
has pointer to object type. \exitnote

\rSec2[std2.range.primitives.cdata]{\tcode{cdata}}
\pnum
The name \tcode{cdata} denotes a customization point
object~(\cxxref{customization.point.object}). The expression
\tcode{\changed{ranges}{::std2}::cdata(E)} for some subexpression \tcode{E} of type \tcode{T}
is expression-equivalent to \tcode{\changed{ranges}{::std2}::data(static_cast<const T\&>(E))}.

\pnum
Use of \tcode{\changed{ranges}{::std2}::cdata(E)} with rvalue \tcode{E} is deprecated.
\enternote This deprecated usage exists so that \tcode{\changed{ranges}{::std2}::cdata(E)}
has behavior consistent with \tcode{\changed{ranges}{::std2}::data(E)} when \tcode{E} is
an rvalue. \exitnote

\pnum
\enternote Whenever \tcode{\changed{ranges}{::std2}::cdata(E)} is a valid expression, it
has pointer to object type. \exitnote

\rSec1[std2.ranges.requirements]{Range requirements}

\rSec2[std2.ranges.requirements.general]{General}

\pnum
Ranges are an abstraction of containers that allow a \Cpp program to
operate on elements of data structures uniformly. It their simplest form, a
range object is one on which one can call \tcode{begin} and
\tcode{end} to get an iterator~(\ref{std2.iterators.iterator}) and a
sentinel~(\ref{std2.iterators.sentinel}). To be able to construct
template algorithms and range adaptors that work correctly and efficiently on
different types of sequences, the library formalizes not just the interfaces but
also the semantics and complexity assumptions of ranges.

\pnum
This document defines three fundamental categories of ranges
based on the syntax and semantics supported by each: \techterm{range},
\techterm{sized range} and \techterm{view}, as shown in
Table~\ref{tab:ranges.relations}.

\begin{floattable}{Relations among range categories}{tab:ranges.relations}
  {lll}
  \topline
  \textbf{Sized Range}  &               &                   \\
                        & $\searrow$    &                   \\
                        &               &  \textbf{Range}   \\
                        & $\nearrow$    &                   \\
  \textbf{View}         &               &                   \\
\end{floattable}

\pnum
The \tcode{Range} concept requires only that \tcode{begin} and \tcode{end}
return an iterator and a sentinel. The \tcode{SizedRange} concept refines \tcode{Range}
with the requirement that the number of elements in the range can be determined
in constant time using the \tcode{size} function. The \tcode{View} concept
specifies requirements on a \tcode{Range} type
with constant-time copy and assign operations.

\pnum
In addition to the three fundamental range categories, this document defines
a number of convenience refinements of \tcode{Range} that group together requirements
that appear often in the concepts and algorithms.
\changed{\textit{Bounded ranges}}{\techterm{Common ranges}} are ranges for which
\tcode{begin} and \tcode{end} return objects of the
same type. \techterm{Random access ranges} are ranges for which
\tcode{begin} returns a type that satisfies
\tcode{RandomAccessIterator}~(\ref{std2.iterators.random.access}). The range
categories \techterm{bidirectional ranges},
\techterm{forward ranges},
\techterm{input ranges}, and
\techterm{output ranges} are defined similarly.

\rSec2[std2.ranges.range]{Ranges}

\pnum
The \tcode{Range} concept defines the requirements of a type that allows
iteration over its elements by providing a \tcode{begin} iterator and an
\tcode{end} sentinel.
\enternote Most algorithms requiring this concept simply forward to an
\tcode{Iterator}-based algorithm by calling \tcode{begin} and \tcode{end}. \exitnote

\begin{itemdecl}
template <class T>
concept @\removed{bool}@ Range =
  requires(T&& t) {
    @\changed{ranges}{::std2}@::begin(t); // not necessarily equality-preserving (see below)
    @\changed{ranges}{::std2}@::end(t);
  };
\end{itemdecl}

\begin{itemdescr}

\pnum
Given an lvalue \tcode{t} of type \tcode{remove_reference_t<T>}, \tcode{Range<T>} is satisfied
only if

\begin{itemize}
\item \range{begin(t)}{end(t)} denotes a range.

\item Both \tcode{begin(t)} and \tcode{end(t)} are amortized constant time
and non-modifying. \enternote \tcode{begin(t)} and \tcode{end(t)} do not require
implicit expression variations~(\cxxref{concepts.lib.general.equality}). \exitnote

\item If \tcode{iterator_t<T>} satisfies \tcode{ForwardIterator},
\tcode{begin(t)} is equality preserving.
\end{itemize}
\end{itemdescr}

\pnum \enternote
Equality preservation of both \tcode{begin} and \tcode{end} enables passing a \tcode{Range}
whose iterator type satisfies \tcode{ForwardIterator}
to multiple algorithms and
making multiple passes over the range by repeated calls to \tcode{begin} and \tcode{end}.
Since \tcode{begin} is not required to be equality preserving when the return type does
not satisfy \tcode{ForwardIterator}, repeated calls might not return equal values or
might not be well-defined; \tcode{begin} should be called at most once for such a range.
\exitnote

\rSec2[std2.ranges.sized]{Sized ranges}

\pnum
The \tcode{SizedRange} concept specifies the requirements
of a \tcode{Range} type that knows its size in constant time with the
\tcode{size} function.

\begin{itemdecl}
template <class T>
concept @\removed{bool}@ SizedRange =
  Range<T> &&
  !disable_sized_range<remove_cv_t<remove_reference_t<T>>> &&
  requires(T& t) {
    { @\changed{ranges}{::std2}@::size(t) } -> ConvertibleTo<difference_type_t<iterator_t<T>>>;
  };
\end{itemdecl}

\begin{itemdescr}
\pnum
Given an lvalue \tcode{t} of type \tcode{remove_reference_t<T>}, \tcode{SizedRange<T>} is satisfied only if:

\begin{itemize}
\item \tcode{\changed{ranges}{::std2}::size(t)} is \bigoh{1}, does not modify \tcode{t}, and is equal
to \tcode{\changed{ranges}{::std2}::distance(t)}.

\item If \tcode{iterator_t<T>} satisfies \tcode{ForwardIterator},
\tcode{size(t)} is well-defined regardless of the evaluation of
\tcode{begin(t)}. \enternote \tcode{size(t)} is otherwise not required be
well-defined after evaluating \tcode{begin(t)}. For a \tcode{SizedRange}
whose iterator type does not model \tcode{ForwardIterator}, for
example, \tcode{size(t)} might only be well-defined if evaluated before
the first call to \tcode{begin(t)}. \exitnote
\end{itemize}

\pnum
\enternote The \tcode{disable_sized_range} predicate provides a mechanism to enable use
of range types with the library that meet the syntactic requirements but do
not in fact satisfy \tcode{SizedRange}. A program that instantiates a library template
that requires a \tcode{Range} with such a range type \tcode{R} is ill-formed with no
diagnostic required unless
\tcode{disable_sized_range<remove_cv_t<remove_reference_t<R>{>}{>}} evaluates
to \tcode{true}~(\ref{std2.structure.requirements}). \exitnote
\end{itemdescr}

\rSec2[std2.ranges.view]{Views}

\pnum
The \tcode{View} concept specifies the requirements of a
\tcode{Range} type that has constant time copy, move and assignment operators; that
is, the cost of these operations is not proportional to the number of elements in
the \tcode{View}.

\pnum
\enterexample
Examples of \tcode{View}s are:

\begin{itemize}
\item A \tcode{Range} type that wraps a pair of iterators.

\item A \tcode{Range} type that holds its elements by \tcode{shared_ptr}
and shares ownership with all its copies.

\item A \tcode{Range} type that generates its elements on demand.
\end{itemize}

A container~(\cxxref{containers}) is not a \tcode{View} since copying the
container copies the elements, which cannot be done in constant time.
\exitexample

\begin{itemdecl}
template <class T>
constexpr bool @\placeholder{view-predicate}@ // \expos
  = @\seebelow@;

template <class T>
concept @\removed{bool}@ View =
  Range<T> &&
  Semiregular<T> &&
  @\placeholder{view-predicate}@<T>;
\end{itemdecl}

\begin{itemdescr}
\pnum
Since the difference between \tcode{Range} and \tcode{View} is largely semantic, the
two are differentiated with the help of the \tcode{enable_view}
trait. Users may specialize \tcode{enable_view}
to derive from \tcode{true_type} or \tcode{false_type}.

\pnum
For a type \tcode{T}, the value of \tcode{\placeholder{view-predicate}<T>} shall be:
\begin{itemize}
\item If \tcode{enable_view<T>} has a member type \tcode{type}, \tcode{enable_view<T>::type::value};
\item Otherwise, if \tcode{T} is derived from \tcode{view_base}, \tcode{true};
\item Otherwise, if \tcode{T} is an instantiation of class template
\tcode{initializer_list}~(\cxxref{support.initlist}),
\tcode{set}~(\cxxref{set}),
\tcode{multiset}~(\cxxref{multiset}),
\tcode{unordered_set}~(\cxxref{unord.set}), or
\tcode{unordered_multiset}~(\cxxref{unord.multiset}), \tcode{false};
\item Otherwise, if both \tcode{T} and \tcode{const T} satisfy \tcode{Range} and
\tcode{reference_t<iterator_t<T>{>}} is not the same type as \tcode{reference_t<iterator_t<const T>{>}},
\tcode{false}; \enternote Deep \tcode{const}-ness implies element ownership, whereas shallow \tcode{const}-ness
implies reference semantics. \exitnote
\item Otherwise, \tcode{true}.
\end{itemize}
\end{itemdescr}

\rSec2[std2.ranges.common]{Common ranges}

\ednote{We suggest changing ``\tcode{BoundedRange}'' to ``\tcode{CommonRange}''. The authors believe
this is a better name than ``\tcode{ClassicRange}'', which LEWG weakly preferred. The reason is
that the iterator and sentinel of a Common range have the same type in \textit{common}.
A non-Common range can be turned into a Common range with the help of \tcode{common_iterator}.
P0789R2 ``Range Adaptors and Utilties'' will be proposing a \tcode{view::common} adaptor that
does precisely that.}

\pnum
The \changed{\tcode{BoundedRange}}{\tcode{CommonRange}} concept specifies requirements
of a \tcode{Range} type for which \tcode{begin} and \tcode{end} return objects of
the same type. \enternote The standard containers~(\cxxref{containers})
satisfy \changed{\tcode{BoundedRange}}{\tcode{CommonRange}}.\exitnote

\begin{codeblock}
template <class T>
concept @\removed{bool}@ @\changed{BoundedRange}{CommonRange}@ =
  Range<T> && Same<iterator_t<T>, sentinel_t<T>>;
\end{codeblock}

\rSec2[std2.ranges.input]{Input ranges}

\pnum
The \tcode{InputRange} concept specifies requirements of
a \tcode{Range} type for which \tcode{begin} returns a type
that satisfies \tcode{InputIterator}~(\ref{std2.iterators.input}).

\begin{codeblock}
template <class T>
concept @\removed{bool}@ InputRange =
  Range<T> && InputIterator<iterator_t<T>>;
\end{codeblock}

\rSec2[std2.ranges.output]{Output ranges}

\pnum
The \tcode{OutputRange} concept specifies requirements of
a \tcode{Range} type for which \tcode{begin} returns a type that satisfies
\tcode{OutputIterator}~(\ref{std2.iterators.output}).

\begin{codeblock}
template <class R, class T>
concept @\removed{bool}@ OutputRange =
  Range<R> && OutputIterator<iterator_t<R>, T>;
\end{codeblock}

\rSec2[std2.ranges.forward]{Forward ranges}

\pnum
The \tcode{ForwardRange} concept specifies requirements of an
\tcode{InputRange} type for which \tcode{begin} returns a type that satisfies
\tcode{ForwardIterator}~(\ref{std2.iterators.forward}).

\begin{codeblock}
template <class T>
concept @\removed{bool}@ ForwardRange =
  InputRange<T> && ForwardIterator<iterator_t<T>>;
\end{codeblock}

\rSec2[std2.ranges.bidirectional]{Bidirectional ranges}

\pnum
The \tcode{BidirectionalRange} concept specifies requirements of a
\tcode{ForwardRange} type for which \tcode{begin} returns a type that satisfies
\tcode{BidirectionalIterator}~(\ref{std2.iterators.bidirectional}).

\begin{codeblock}
template <class T>
concept @\removed{bool}@ BidirectionalRange =
  ForwardRange<T> && BidirectionalIterator<iterator_t<T>>;
\end{codeblock}

\rSec2[std2.ranges.random.access]{Random access ranges}

\pnum
The \tcode{RandomAccessRange} concept specifies requirements of a
\tcode{BidirectionalRange} type for which \tcode{begin} returns a type that satisfies
\tcode{RandomAccessIterator}~(\ref{std2.iterators.random.access}).

\begin{codeblock}
template <class T>
concept @\removed{bool}@ RandomAccessRange =
  BidirectionalRange<T> && RandomAccessIterator<iterator_t<T>>;
\end{codeblock}

\ednote{\tcode{dangling} moved here from the ``Iterators library''.}

\rSec1[std2.dangling.wrappers]{Dangling wrapper}

\rSec2[std2.dangling.wrap]{Class template \tcode{dangling}}

\pnum
\indexlibrary{\idxcode{dangling}}%
Class template \tcode{dangling} is a wrapper for an object that refers to another object whose
lifetime may have ended. It is used by algorithms that accept rvalue ranges and return iterators.

\begin{codeblock}
namespace @\changed{std \{ namespace experimental \{ namespace ranges}{std2}@ { inline namespace v1 {
  template <CopyConstructible T>
  class dangling {
  public:
    constexpr dangling() requires DefaultConstructible<T>;
    constexpr dangling(T t);
    constexpr T get_unsafe() const;
  private:
    T value; // \expos
  };

  template <Range R>
  using safe_iterator_t =
    conditional_t<is_lvalue_reference@\added{_v}@<R>@\removed{::value}@,
      iterator_t<R>,
      dangling<iterator_t<R>>>;
}}@\removed{\}\}}@
\end{codeblock}

\rSec3[std2.dangling.wrap.ops]{\tcode{dangling} operations}

\rSec4[std2.dangling.wrap.op.const]{\tcode{dangling} constructors}

\indexlibrary{\idxcode{dangling}!\idxcode{dangling}}%
\begin{itemdecl}
constexpr dangling() requires DefaultConstructible<T>;
\end{itemdecl}

\begin{itemdescr}
\pnum
\effects Constructs a \tcode{dangling}, value-initializing \tcode{value}.
\end{itemdescr}

\indexlibrary{\idxcode{dangling}!\idxcode{dangling}}%
\begin{itemdecl}
constexpr dangling(T t);
\end{itemdecl}

\begin{itemdescr}
\pnum
\effects Constructs a \tcode{dangling}, initializing \tcode{value} with \tcode{t}.
\end{itemdescr}

\rSec4[std2.dangling.wrap.op.get]{\tcode{dangling::get_unsafe}}

\indexlibrary{\idxcode{get_unsafe}!\idxcode{dangling}}%
\indexlibrary{\idxcode{dangling}!\idxcode{get_unsafe}}%
\begin{itemdecl}
constexpr T get_unsafe() const;
\end{itemdecl}

\begin{itemdescr}
\pnum
\returns \tcode{value}.
\end{itemdescr}

