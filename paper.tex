\documentclass{article}
\usepackage[T1]{fontenc}
\usepackage{color}
\usepackage{cite}
\usepackage[pdftex, bookmarks=true]{hyperref}
\usepackage{amsmath}
\usepackage{listings}
\usepackage{tikz}
\usetikzlibrary{matrix}
\usepackage{pgfplots}
\usepgfplotslibrary{groupplots}
\usepackage[margin=0.75in]{geometry}

% Make a draft watermark.
% \usepackage{draftwatermark}
% \SetWatermarkFontSize{40pt}
% \SetWatermarkScale{4.0}

% TODO: Use bibtex or some better bibliography management
% tool.



% Define the speculative C++1y language.
\lstdefinelanguage{C++1y}{alsolanguage=C++,
                          escapechar=@,
                          breakatwhitespace=true,
                          morekeywords = {alignof, 
                                          decltype, 
                                          concept, 
                                          axiom, 
                                          requires, 
                                          property}}

% Program output
\lstdefinelanguage{Output}{}

% EBNF
\lstdefinelanguage{Ebnf}
{
  escapechar=@,
  basicstyle=\itshape\small
}


% Default formatting for listings.
%
% TODO: Define this as a style and make the code and program macros refer
% to the style.
\lstset{language=C++1y,
        basicstyle=\ttfamily\small,
        keywordstyle=\bfseries\color[rgb]{0,0,1},
        stringstyle=,
        xleftmargin=1em,
        showstringspaces=false,
        commentstyle=\rmfamily\itshape,
        columns=flexible,
        keepspaces=true,
        texcl=true}




% Define formatting for code.
\newcommand{\code}[1]{\lstinline @#1@}

% Define an environment for C++ programs.
\lstnewenvironment{program}{\lstset{language=C++1y}}{}

% Define an environment for program output.
\lstnewenvironment{progout}{\lstset{language=Output}}{}

% Define an environment for EBNF Grammars
\lstnewenvironment{ebnf}{\lstset{language=Ebnf}}{}

% Define formatting for indented blocks of text
\newcommand{\blockindent}[1]{\hangindent=#1\setlength{\parindent}{#1}\setlength{\parskip}{1em plus0.2em minus0.2em}}

% Define formatting for indented blocks of text
\newcommand{\blockunindent}{\hangindent=0em\setlength{\parindent}{0em}\setlength{\parskip}{0em}}


\begin{document}

\title{\textbf{\Large \code{find_backward}}}
\date{}
{\let\newpage\relax\maketitle}

\noindent\textbf{Document Number:} P1223R1\\
\textbf{Date:} 2018-10-02\\
\textbf{Reply to:} Zach Laine whatwasthataddress@gmail.com\\
\textbf{Audience:} LEWG

\section{Revisions}

\subsection{Changes from R0}

\begin{itemize}
  \item Base synopsis on The One Ranges Proposal (P0896R4).
  \item Drop \code{std}-namespace overloads.
  \item Drop \code{find_not()} and \code{find_not_backward()}.
\end{itemize}

\section{Introduction}

\label{sec:intro}

Sometimes you need to search backward.  This is often awkward to do with
\code{find} and \code{make_reverse_iterator}.  We should have first-class
algorithms to turn this:

\lstinputlisting[language=C++, firstline=3, lastline=7]{snippets.cpp}

into this:

\lstinputlisting[language=C++, firstline=41, lastline=42]{snippets.cpp}

\section{Motivation and Scope}

Consider how finding the last element that is equal to `x` in a range is
typically done (for all the examples below, we assume a valid range of
elements \code{[first, last)}, and an iterator \code{it} within that range):

\lstinputlisting[language=C++, firstline=3, lastline=7]{snippets.cpp}

Raw loops are icky though.  Perhaps we should do a bit of extra work to allow
the use of \code{find()}:

\lstinputlisting[language=C++, firstline=12, lastline=15]{snippets.cpp}

That seems nicer in that there is no raw loop, but it requires an unpleasant
amount of typing (and an associated lack of clarity).

Consider this instead:

\lstinputlisting[language=C++, firstline=41, lastline=42]{snippets.cpp}

That's better!  It's a lot less verbose.

Let's consider for a moment the lack of clarity of the
\code{make_reverse_iterator()} code.  In a typical use of \code{find()}, I
search forward from the element I start from, including the element itself:

\lstinputlisting[language=C++, firstline=20, lastline=20]{snippets.cpp}

However, using finding in reverse in the middle of a range leaves out the
element pointed to by the current iterator:

\lstinputlisting[language=C++, firstline=25, lastline=28]{snippets.cpp}

That leads to code like this:

\lstinputlisting[language=C++, firstline=33, lastline=36]{snippets.cpp}

Though this looks like an off-by-one error. is is correct.  Moreover, even
though the use of \code{next()} is correct, it gets lost in noise of the rest
of the code, since it is so verbose.  Use \code{find_backward()} makes things
clearer:

\lstinputlisting[language=C++, firstline=47, lastline=51]{snippets.cpp}

The use of \code{next()} may at first appear like a mistake, until the reader
takes a moment to think things through.  In the \code{reverse_iterator}
version, this correctness is a lot harder to readily grasp.

\section{Proposed Design}

\subsection{Design}

This paper proposes to introduce iterator-based and range-based overloads of
the functions \code{find_backward()}, \code{find_not_backward()},
\code{find_if_backward()}, \code{find_if_not_backward()}, and
\code{find_not()}.  The following synopsis has interface details.  Note that
the iterator-based *\code{_backward} overloads in namespace \code{ranges} do
not take an iterator-sentinel pair; this is not suitable for an algorithm that
operates in reverse.  \code{find_not()}, being the only forward-operating
algorithm proposed, does have a sentinel-accepting overload.

\subsubsection{\code{flat_set} Synopsis}

\lstinputlisting[language=C++]{synopsis.hpp}


\section{Acknowledgements}

Thanks to Alisdair Meredith and Marshall Clow for encouraging this submission.


\end{document}
