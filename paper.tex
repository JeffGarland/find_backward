%%--------------------------------------------------
%% basics
% \documentclass[letterpaper,oneside,openany]{memoir}
\documentclass[ebook,10pt,oneside,openany,final]{memoir}
% \includeonly{declarations}

\usepackage[american]
           {babel}        % needed for iso dates
\usepackage[iso,american]
           {isodate}      % use iso format for dates
\usepackage[final]
           {listings}     % code listings
\usepackage{longtable}    % auto-breaking tables
\usepackage{ltcaption}    % fix captions for long tables
\usepackage{booktabs}     % fancy tables
\usepackage{relsize}      % provide relative font size changes
\usepackage{underscore}   % remove special status of '_' in ordinary text
\usepackage{verbatim}     % improved verbatim environment
\usepackage{parskip}      % handle non-indented paragraphs "properly"
\usepackage{array}        % new column definitions for tables
\usepackage[normalem]{ulem}
\usepackage{color}        % define colors for strikeouts and underlines
\usepackage{amsmath}      % additional math symbols
\usepackage{mathrsfs}     % mathscr font
\usepackage{multicol}
\usepackage{xspace}
\usepackage{fixme}
\usepackage{lmodern}
\usepackage{textcomp}     % provide \text{l,r}angle
\usepackage[T1]{fontenc}
\usepackage[pdftex, final]{graphicx}
\usepackage[pdftex,
            pdftitle={find\_last},
            pdfsubject={find\_last},
            pdfcreator={Zach Laine},
            bookmarks=true,
            bookmarksnumbered=true,
            pdfpagelabels=true,
            pdfpagemode=UseOutlines,
            pdfstartview=FitH,
            linktocpage=true,
            colorlinks=true,
            linkcolor=blue,
            plainpages=false
           ]{hyperref}
\usepackage{memhfixc}     % fix interactions between hyperref and memoir

\input{layout}
\input{styles}
\input{macros}
\input{tables}
\input{cxx}

%%--------------------------------------------------
%% fix interaction between hyperref and other
%% commands
\pdfstringdefDisableCommands{\def\smaller#1{#1}}
\pdfstringdefDisableCommands{\def\textbf#1{#1}}
\pdfstringdefDisableCommands{\def\raisebox#1{}}
\pdfstringdefDisableCommands{\def\hspace#1{}}

%%--------------------------------------------------
%% add special hyphenation rules
\hyphenation{tem-plate ex-am-ple in-put-it-er-a-tor name-space name-spaces}

\begin{document}
\chapterstyle{cppstd}
\pagestyle{cpppage}

%%--------------------------------------------------
%% configuration
\input{config}

%%--------------------------------------------------
%% front matter
\frontmatter
\include{front}

%%--------------------------------------------------
%% main body of the document
\mainmatter
\setglobalstyles

\section{Revisions}

\subsection{Changes from R1}

\begin{itemize}
  \item Change \tcode{find_backward()} to \tcode{find_last()}.
  \item Wording.
\end{itemize}

\subsection{Changes from R0}

\begin{itemize}
  \item Base synopsis on The One Ranges Proposal (P0896R4).
  \item Drop \tcode{std}-namespace overloads.
  \item Drop \tcode{find_not()} and \tcode{find_not_backward()}.
\end{itemize}


\section{Motivation and Scope}

Consider how finding the last element that is equal to `x` in a range is
typically done (for all the examples below, we assume a valid range of
elements \tcode{[first, last)}, and an iterator \tcode{it} within that range):

\begin{itemdecl}
    while (it-- != first) {
        if (*it == x) {
            // Use it here...
        }
    }
\end{itemdecl}

Raw loops are icky though.  Perhaps we should do a bit of extra work to allow
the use of \tcode{find()}:

\begin{itemdecl}
    auto rfirst = std::make_reverse_iterator(it);
    auto rlast = std::make_reverse_iterator(first);
    auto it = std::find(rfirst, rlast, x);
    // Use it here...
\end{itemdecl}

That seems nicer in that there is no raw loop, but it requires an unpleasant
amount of typing (and an associated lack of clarity).

Consider this instead:

\begin{itemdecl}
    auto it = std::find_last(first, it, x);
    // Use it here...
\end{itemdecl}

That's better!  It's a lot less verbose.

Let's consider for a moment the lack of clarity of the
\tcode{make_reverse_iterator()} code.  In a typical use of \tcode{find()}, I
search forward from the element I start from, including the element itself:

\begin{itemdecl}
    auto it = std::find(it, last, x); // Includes examination of *it.
\end{itemdecl}

However, using finding in reverse in the middle of a range leaves out the
element pointed to by the current iterator:

\begin{itemdecl}
    auto it = std::find( // Skips *it entirely.
        std::make_reverse_iterator(first),
        std::make_reverse_iterator(it),
        x);
\end{itemdecl}

That leads to code like this:

\begin{itemdecl}
    auto it = std::find( // Includes *it again!
        std::make_reverse_iterator(first),
        std::make_reverse_iterator(std::next(it)),
        x);
\end{itemdecl}

Though this looks like an off-by-one error, is is correct.  Moreover, even
though the use of \tcode{next()} is correct, it gets lost in noise of the rest
of the code, since it is so verbose.  Use \tcode{find_last()} makes things
clearer:

\begin{itemdecl}
    // Search, but don't include *it.
    auto it_1 = std::find_last(first, it, x);

    // Search, and include *it.
    auto it_2 = std::find_last(first, std::next(it), x);
\end{itemdecl}

The use of \tcode{next()} may at first appear like a mistake, until the reader
takes a moment to think things through.  In the \tcode{reverse_iterator}
version, this correctness is a lot harder to readily grasp.

\section{Feature-Test Macro}

In addition to the wording that follows, add a new feature-test macro
\tcode{__cpp_lib_find_last}, with value \tcode{202012}.  This macro should be
defined in \tcode{<algorithm>}.

\newcommand{\indexhdr}[1]{}
\newcommand{\indexlibrarymember}[2]{
}
%!TEX root = std.tex
\setcounter{chapter}{24}
\setcounter{section}{5}
\setcounter{subsection}{1}
\setcounter{subsubsection}{1}

\rSec0[algorithms]{Algorithms library}

\setcounter{section}{3}

\rSec1[algorithm.syn]{Header \tcode{<algorithm>} synopsis}
\indexhdr{algorithm}%

\begin{codeblock}
#include <initializer_list>

namespace std {
  // \ref{alg.nonmodifying}, non-modifying sequence operations

  // \ref{alg.find}, find
  template<class InputIterator, class T>
    constexpr InputIterator find(InputIterator first, InputIterator last,
                                 const T& value);
  template<class ExecutionPolicy, class ForwardIterator, class T>
    ForwardIterator find(ExecutionPolicy&& exec, // see \ref{algorithms.parallel.overloads}
                         ForwardIterator first, ForwardIterator last,
                         const T& value);
  template<class InputIterator, class Predicate>
    constexpr InputIterator find_if(InputIterator first, InputIterator last,
                                    Predicate pred);
  template<class ExecutionPolicy, class ForwardIterator, class Predicate>
    ForwardIterator find_if(ExecutionPolicy&& exec, // see \ref{algorithms.parallel.overloads}
                            ForwardIterator first, ForwardIterator last,
                            Predicate pred);
  template<class InputIterator, class Predicate>
    constexpr InputIterator find_if_not(InputIterator first, InputIterator last,
                                        Predicate pred);
  template<class ExecutionPolicy, class ForwardIterator, class Predicate>
    ForwardIterator find_if_not(ExecutionPolicy&& exec, // see \ref{algorithms.parallel.overloads}
                                ForwardIterator first, ForwardIterator last,
                                Predicate pred);

  namespace ranges {
    template<InputIterator I, Sentinel<I> S, class T, class Proj = identity>
      requires IndirectRelation<ranges::equal_to, projected<I, Proj>, const T*>
      constexpr I find(I first, S last, const T& value, Proj proj = {});
    template<InputRange R, class T, class Proj = identity>
      requires IndirectRelation<ranges::equal_to, projected<iterator_t<R>, Proj>, const T*>
      constexpr safe_iterator_t<R>
        find(R&& r, const T& value, Proj proj = {});
    template<InputIterator I, Sentinel<I> S, class Proj = identity,
             IndirectUnaryPredicate<projected<I, Proj>> Pred>
      constexpr I find_if(I first, S last, Pred pred, Proj proj = {});
    template<InputRange R, class Proj = identity,
             IndirectUnaryPredicate<projected<iterator_t<R>, Proj>> Pred>
      constexpr safe_iterator_t<R>
        find_if(R&& r, Pred pred, Proj proj = {});
    template<InputIterator I, Sentinel<I> S, class Proj = identity,
             IndirectUnaryPredicate<projected<I, Proj>> Pred>
      constexpr I find_if_not(I first, S last, Pred pred, Proj proj = {});
    template<InputRange R, class Proj = identity,
             IndirectUnaryPredicate<projected<iterator_t<R>, Proj>> Pred>
      constexpr safe_iterator_t<R>
        find_if_not(R&& r, Pred pred, Proj proj = {});
  }
\end{codeblock}
\begin{addedblock}
\begin{codeblock}

  // \ref{alg.find.last}, find last
  namespace ranges {
    template<ForwardIterator I, Sentinel<I> S, class T, class Proj = identity>
      requires IndirectRelation<ranges::equal_to, projected<I, Proj>, const T*>
      constexpr I find_last(I first, S last, const T& value, Proj proj = {});
    template<ForwardRange R, class T, class Proj = identity>
      requires IndirectRelation<ranges::equal_to, projected<iterator_t<R>, Proj>, const T*>
      constexpr safe_iterator_t<R>
        find_last(R&& r, const T& value, Proj proj = {});
    template<ForwardIterator I, Sentinel<I> S, class Proj = identity,
             IndirectUnaryPredicate<projected<I, Proj>> Pred>
      constexpr I find_last_if(I first, S last, Pred pred, Proj proj = {});
    template<ForwardRange R, class Proj = identity,
             IndirectUnaryPredicate<projected<iterator_t<R>, Proj>> Pred>
      constexpr safe_iterator_t<R>
        find_last_if(R&& r, Pred pred, Proj proj = {});
    template<ForwardIterator I, Sentinel<I> S, class Proj = identity,
             IndirectUnaryPredicate<projected<I, Proj>> Pred>
      constexpr I find_last_if_not(I first, S last, Pred pred, Proj proj = {});
    template<ForwardRange R, class Proj = identity,
             IndirectUnaryPredicate<projected<iterator_t<R>, Proj>> Pred>
      constexpr safe_iterator_t<R>
        find_last_if_not(R&& r, Pred pred, Proj proj = {});
  }
\end{codeblock}
\end{addedblock}
\begin{codeblock}
}
\end{codeblock}

\rSec1[alg.nonmodifying]{Non-modifying sequence operations}

\setcounter{subsection}{4}
\rSec2[alg.find]{Find}

\indexlibrary{\idxcode{find}}%
\indexlibrary{\idxcode{find_if}}%
\indexlibrary{\idxcode{find_if_not}}%
\begin{itemdecl}
template<class InputIterator, class T>
  constexpr InputIterator find(InputIterator first, InputIterator last,
                               const T& value);
template<class ExecutionPolicy, class ForwardIterator, class T>
  ForwardIterator find(ExecutionPolicy&& exec, ForwardIterator first, ForwardIterator last,
                       const T& value);

template<class InputIterator, class Predicate>
  constexpr InputIterator find_if(InputIterator first, InputIterator last,
                                  Predicate pred);
template<class ExecutionPolicy, class ForwardIterator, class Predicate>
  ForwardIterator find_if(ExecutionPolicy&& exec, ForwardIterator first, ForwardIterator last,
                          Predicate pred);

template<class InputIterator, class Predicate>
  constexpr InputIterator find_if_not(InputIterator first, InputIterator last,
                                      Predicate pred);
template<class ExecutionPolicy, class ForwardIterator, class Predicate>
  ForwardIterator find_if_not(ExecutionPolicy&& exec,
                              ForwardIterator first, ForwardIterator last,
                              Predicate pred);

template<InputIterator I, Sentinel<I> S, class T, class Proj = identity>
 requires IndirectRelation<ranges::equal_to, projected<I, Proj>, const T*>
 constexpr I ranges::find(I first, S last, const T& value, Proj proj = {});
template<InputRange R, class T, class Proj = identity>
 requires IndirectRelation<ranges::equal_to, projected<iterator_t<R>, Proj>, const T*>
 constexpr safe_iterator_t<R>
   ranges::find(R&& r, const T& value, Proj proj = {});
template<InputIterator I, Sentinel<I> S, class Proj = identity,
        IndirectUnaryPredicate<projected<I, Proj>> Pred>
 constexpr I ranges::find_if(I first, S last, Pred pred, Proj proj = {});
template<InputRange R, class Proj = identity,
        IndirectUnaryPredicate<projected<iterator_t<R>, Proj>> Pred>
 constexpr safe_iterator_t<R>
   ranges::find_if(R&& r, Pred pred, Proj proj = {});
template<InputIterator I, Sentinel<I> S, class Proj = identity,
        IndirectUnaryPredicate<projected<I, Proj>> Pred>
 constexpr I ranges::find_if_not(I first, S last, Pred pred, Proj proj = {});
template<InputRange R, class Proj = identity,
        IndirectUnaryPredicate<projected<iterator_t<R>, Proj>> Pred>
 constexpr safe_iterator_t<R>
   ranges::find_if_not(R&& r, Pred pred, Proj proj = {});
\end{itemdecl}

\begin{itemdescr}
\pnum
Let $E$ be:
\begin{itemize}
\item \tcode{*i == value} for \tcode{find},
\item \tcode{pred(*i) != false} for \tcode{find_if},
\item \tcode{pred(*i) == false} for \tcode{find_if_not},
\item \tcode{invoke(proj, *i) == value} for \tcode{ranges::find},
\item \tcode{invoke(pred, invoke(proj, *i)) != false} for \tcode{ranges::find_if},
\item \tcode{invoke(pred, invoke(proj, *i)) == false} for \tcode{ranges::find_if_not}.
\end{itemize}

\pnum
\returns
The first iterator \tcode{i} in the range \range{first}{last}
for which $E$ is \tcode{true}.
Returns \tcode{last} if no such iterator is found.

\pnum
\complexity
At most \tcode{last - first} applications
of the corresponding predicate and any projection.
\end{itemdescr}
\begin{addedblock}

\rSec2[alg.find.last]{Find last}

\indexlibrary{\idxcode{find_last}}%
\indexlibrary{\idxcode{find_last_if}}%
\indexlibrary{\idxcode{find_last_if_not}}%
\begin{itemdecl}
template<ForwardIterator I, Sentinel<I> S, class T, class Proj = identity>
 requires IndirectRelation<ranges::equal_to, projected<I, Proj>, const T*>
 constexpr I ranges::find_last(I first, S last, const T& value, Proj proj = {});
template<ForwardRange R, class T, class Proj = identity>
 requires IndirectRelation<ranges::equal_to, projected<iterator_t<R>, Proj>, const T*>
 constexpr safe_iterator_t<R>
   ranges::find_last(R&& r, const T& value, Proj proj = {});
template<ForwardIterator I, Sentinel<I> S, class Proj = identity,
        IndirectUnaryPredicate<projected<I, Proj>> Pred>
 constexpr I ranges::find_last_if(I first, S last, Pred pred, Proj proj = {});
template<ForwardRange R, class Proj = identity,
        IndirectUnaryPredicate<projected<iterator_t<R>, Proj>> Pred>
 constexpr safe_iterator_t<R>
   ranges::find_last_if(R&& r, Pred pred, Proj proj = {});
template<ForwardIterator I, Sentinel<I> S, class Proj = identity,
        IndirectUnaryPredicate<projected<I, Proj>> Pred>
 constexpr I ranges::find_last_if_not(I first, S last, Pred pred, Proj proj = {});
template<ForwardRange R, class Proj = identity,
        IndirectUnaryPredicate<projected<iterator_t<R>, Proj>> Pred>
 constexpr safe_iterator_t<R>
   ranges::find_last_if_not(R&& r, Pred pred, Proj proj = {});
\end{itemdecl}

\begin{itemdescr}
\pnum
Let $E$ be:
\begin{itemize}
\item \tcode{invoke(proj, *i) == value} for \tcode{ranges::find_last},
\item \tcode{invoke(pred, invoke(proj, *i)) != false} for \tcode{ranges::find_last_if},
\item \tcode{invoke(pred, invoke(proj, *i)) == false} for \tcode{ranges::find_last_if_not}.
\end{itemize}

\pnum
\returns
The last iterator \tcode{i} in the range \range{first}{last}
for which $E$ is \tcode{true}.
Returns \tcode{last} if no such iterator is found.

\pnum
\complexity
At most \tcode{last - first} applications
of the corresponding predicate and any projection.
\end{itemdescr}
\end{addedblock}


%%--------------------------------------------------
%!TEX root = stl2-ts.tex
\section{Acknowledgements}

Thanks to Alisdair Meredith and Marshall Clow for encouraging this submission.

Thanks to Jeff Garland for helping sheparding the paper through final LEWG and LWG reviews.


%%--------------------------------------------------
%% End of document
\end{document}
