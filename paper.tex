%%--------------------------------------------------
%% basics
% \documentclass[letterpaper,oneside,openany]{memoir}
\documentclass[ebook,10pt,oneside,openany,final]{memoir}
% \includeonly{declarations}

\usepackage[american]
           {babel}        % needed for iso dates
\usepackage[iso,american]
           {isodate}      % use iso format for dates
\usepackage[final]
           {listings}     % code listings
\usepackage{longtable}    % auto-breaking tables
\usepackage{ltcaption}    % fix captions for long tables
\usepackage{booktabs}     % fancy tables
\usepackage{relsize}      % provide relative font size changes
\usepackage{underscore}   % remove special status of '_' in ordinary text
\usepackage{verbatim}     % improved verbatim environment
\usepackage{parskip}      % handle non-indented paragraphs "properly"
\usepackage{array}        % new column definitions for tables
\usepackage[normalem]{ulem}
\usepackage{color}        % define colors for strikeouts and underlines
\usepackage{amsmath}      % additional math symbols
\usepackage{mathrsfs}     % mathscr font
\usepackage{multicol}
\usepackage{xspace}
\usepackage{fixme}
\usepackage{lmodern}
\usepackage{textcomp}     % provide \text{l,r}angle
\usepackage[T1]{fontenc}
\usepackage[pdftex, final]{graphicx}
\usepackage[pdftex,
            pdftitle={find\_last},
            pdfsubject={find\_last},
            pdfcreator={Zach Laine},
            bookmarks=true,
            bookmarksnumbered=true,
            pdfpagelabels=true,
            pdfpagemode=UseOutlines,
            pdfstartview=FitH,
            linktocpage=true,
            colorlinks=true,
            linkcolor=blue,
            plainpages=false
           ]{hyperref}
\usepackage{memhfixc}     % fix interactions between hyperref and memoir

%!TEX root = D0896.tex
%% layout.tex -- set overall page appearance

%%--------------------------------------------------
%%  set page size, type block size, type block position

\setstocksize{11in}{8.5in}
\settrimmedsize{11in}{8.5in}{*}
\setlrmarginsandblock{1in}{1in}{*}
\setulmarginsandblock{1in}{*}{1.618}

%%--------------------------------------------------
%%  set header and footer positions and sizes

\setheadfoot{\onelineskip}{2\onelineskip}
\setheaderspaces{*}{2\onelineskip}{*}

%%--------------------------------------------------
%%  make miscellaneous adjustments, then finish the layout
\setmarginnotes{7pt}{7pt}{0pt}
\checkandfixthelayout

%%--------------------------------------------------
%% Paragraph and bullet numbering

\newcounter{Paras}
\counterwithin{Paras}{chapter}
\counterwithin{Paras}{section}
\counterwithin{Paras}{subsection}
\counterwithin{Paras}{subsubsection}
\counterwithin{Paras}{paragraph}
\counterwithin{Paras}{subparagraph}

\newcounter{Bullets1}[Paras]
\newcounter{Bullets2}[Bullets1]
\newcounter{Bullets3}[Bullets2]
\newcounter{Bullets4}[Bullets3]

\makeatletter
\newcommand{\parabullnum}[2]{%
\stepcounter{#1}%
\noindent\makebox[0pt][l]{\makebox[#2][r]{%
\scriptsize\raisebox{.7ex}%
{%
\ifnum \value{Paras}>0
\ifnum \value{Bullets1}>0 (\fi%
                          \arabic{Paras}%
\ifnum \value{Bullets1}>0 .\arabic{Bullets1}%
\ifnum \value{Bullets2}>0 .\arabic{Bullets2}%
\ifnum \value{Bullets3}>0 .\arabic{Bullets3}%
\fi\fi\fi%
\ifnum \value{Bullets1}>0 )\fi%
\fi%
}%
\hspace{\@totalleftmargin}\quad%
}}}
\makeatother

\def\pnum{\parabullnum{Paras}{0pt}}

% Leave more room for section numbers in TOC
\cftsetindents{section}{1.5em}{3.0em}

% For compatibility only.  We no longer need this environment.
\newenvironment{paras}{}{}

%!TEX root = D0896.tex
%% styles.tex -- set styles for:
%     chapters
%     pages
%     footnotes

%%--------------------------------------------------
%%  create chapter style

\makechapterstyle{cppstd}{%
  \renewcommand{\beforechapskip}{\onelineskip}
  \renewcommand{\afterchapskip}{\onelineskip}
  \renewcommand{\chapternamenum}{}
  \renewcommand{\chapnamefont}{\chaptitlefont}
  \renewcommand{\chapnumfont}{\chaptitlefont}
  \renewcommand{\printchapternum}{\chapnumfont\thechapter\quad}
  \renewcommand{\afterchapternum}{}
}

%%--------------------------------------------------
%%  create page styles

\makepagestyle{cpppage}
\makeevenhead{cpppage}{\copyright\,\textsc{ISO/IEC}}{}{\textbf{\docno}}
\makeoddhead{cpppage}{\copyright\,\textsc{ISO/IEC}}{}{\textbf{\docno}}
%\makeevenfoot{cpppage}{\leftmark}{}{\thepage}
%\makeoddfoot{cpppage}{\leftmark}{}{\thepage}
\makeevenfoot{cpppage}{}{}{\thepage}
\makeoddfoot{cpppage}{}{}{\thepage}

\makeatletter
\makepsmarks{cpppage}{%
  \let\@mkboth\markboth
  \def\chaptermark##1{\markboth{##1}{##1}}%
  \def\sectionmark##1{\markboth{%
    \ifnum \c@secnumdepth>\z@
      \textsection\space\thesection
    \fi
    }{\rightmark}}%
  \def\subsectionmark##1{\markboth{%
    \ifnum \c@secnumdepth>\z@
      \textsection\space\thesubsection
    \fi
    }{\rightmark}}%
  \def\subsubsectionmark##1{\markboth{%
    \ifnum \c@secnumdepth>\z@
      \textsection\space\thesubsubsection
    \fi
    }{\rightmark}}%
  \def\paragraphmark##1{\markboth{%
    \ifnum \c@secnumdepth>\z@
      \textsection\space\theparagraph
    \fi
    }{\rightmark}}}
\makeatother

\aliaspagestyle{chapter}{cpppage}

%%--------------------------------------------------
%%  set heading styles for main matter
\newcommand{\beforeskip}{-.7\onelineskip plus -1ex}
\newcommand{\afterskip}{.3\onelineskip minus .2ex}

\setbeforesecskip{\beforeskip}
\setsecindent{0pt}
\setsecheadstyle{\large\bfseries\raggedright}
\setaftersecskip{\afterskip}

\setbeforesubsecskip{\beforeskip}
\setsubsecindent{0pt}
\setsubsecheadstyle{\large\bfseries\raggedright}
\setaftersubsecskip{\afterskip}

\setbeforesubsubsecskip{\beforeskip}
\setsubsubsecindent{0pt}
\setsubsubsecheadstyle{\normalsize\bfseries\raggedright}
\setaftersubsubsecskip{\afterskip}

\setbeforeparaskip{\beforeskip}
\setparaindent{0pt}
\setparaheadstyle{\normalsize\bfseries\raggedright}
\setafterparaskip{\afterskip}

\setbeforesubparaskip{\beforeskip}
\setsubparaindent{0pt}
\setsubparaheadstyle{\normalsize\bfseries\raggedright}
\setaftersubparaskip{\afterskip}

%%--------------------------------------------------
% set heading style for annexes
\newcommand{\Annex}[3]{\chapter[#2]{(#3)\protect\\#2\hfill[#1]}\relax\label{#1}}
\newcommand{\infannex}[2]{\Annex{#1}{#2}{informative}}
\newcommand{\normannex}[2]{\Annex{#1}{#2}{normative}}

%%--------------------------------------------------
%%  set footnote style
\footmarkstyle{\smaller#1) }

%%--------------------------------------------------
% set style for main text
\setlength{\parindent}{0pt}
\setlength{\parskip}{1ex}
\setlength{\partopsep}{-1.5ex}

%%--------------------------------------------------
%%  set caption style and delimiter
\captionstyle{\centering}
\captiondelim{ --- }
% override longtable's caption delimiter to match
\makeatletter
\def\LT@makecaption#1#2#3{%
  \LT@mcol\LT@cols c{\hbox to\z@{\hss\parbox[t]\LTcapwidth{%
    \sbox\@tempboxa{#1{#2 --- }#3}%
    \ifdim\wd\@tempboxa>\hsize
      #1{#2 --- }#3%
    \else
      \hbox to\hsize{\hfil\box\@tempboxa\hfil}%
    \fi
    \endgraf\vskip\baselineskip}%
  \hss}}}
\makeatother

%%--------------------------------------------------
%% set global styles that get reset by \mainmatter
\newcommand{\setglobalstyles}{
  \counterwithout{footnote}{chapter}
  \counterwithout{table}{chapter}
  \counterwithout{figure}{chapter}
  \renewcommand{\chaptername}{}
  \renewcommand{\appendixname}{Annex }
}

%%--------------------------------------------------
%% change list item markers to number and em-dash

\renewcommand{\labelitemi}{---\parabullnum{Bullets1}{\labelsep}}
\renewcommand{\labelitemii}{---\parabullnum{Bullets2}{\labelsep}}
\renewcommand{\labelitemiii}{---\parabullnum{Bullets3}{\labelsep}}
\renewcommand{\labelitemiv}{---\parabullnum{Bullets4}{\labelsep}}

%%--------------------------------------------------
%% set section numbering limit, toc limit
\maxsecnumdepth{subparagraph}
\setcounter{tocdepth}{1}



% Define formatting for code.
\newcommand{\code}[1]{\lstinline @#1@}

% Define an environment for C++ programs.
\lstnewenvironment{program}{\lstset{language=C++1y}}{}

% Define an environment for program output.
\lstnewenvironment{progout}{\lstset{language=Output}}{}

% Define an environment for EBNF Grammars
\lstnewenvironment{ebnf}{\lstset{language=Ebnf}}{}

% Define formatting for indented blocks of text
\newcommand{\blockindent}[1]{\hangindent=#1\setlength{\parindent}{#1}\setlength{\parskip}{1em plus0.2em minus0.2em}}

% Define formatting for indented blocks of text
\newcommand{\blockunindent}{\hangindent=0em\setlength{\parindent}{0em}\setlength{\parskip}{0em}}

%!TEX root = std.tex
% Definitions of table environments

%%--------------------------------------------------
%% Table environments

% Set parameters for floating tables
\setcounter{totalnumber}{10}

% Base definitions for tables
\newenvironment{TableBase}
{
 \renewcommand{\tcode}[1]{\CodeStylex{##1}}
 \newcommand{\topline}{\hline}
 \newcommand{\capsep}{\hline\hline}
 \newcommand{\rowsep}{\hline}
 \newcommand{\bottomline}{\hline}

%% vertical alignment
 \newcommand{\rb}[1]{\raisebox{1.5ex}[0pt]{##1}}	% move argument up half a row

%% header helpers
 \newcommand{\hdstyle}[1]{\textbf{##1}}				% set header style
 \newcommand{\Head}[3]{\multicolumn{##1}{##2}{\hdstyle{##3}}}	% add title spanning multiple columns
 \newcommand{\lhdrx}[2]{\Head{##1}{|c}{##2}}		% set header for left column spanning #1 columns
 \newcommand{\chdrx}[2]{\Head{##1}{c}{##2}}			% set header for center column spanning #1 columns
 \newcommand{\rhdrx}[2]{\Head{##1}{c|}{##2}}		% set header for right column spanning #1 columns
 \newcommand{\ohdrx}[2]{\Head{##1}{|c|}{##2}}		% set header for only column spanning #1 columns
 \newcommand{\lhdr}[1]{\lhdrx{1}{##1}}				% set header for single left column
 \newcommand{\chdr}[1]{\chdrx{1}{##1}}				% set header for single center column
 \newcommand{\rhdr}[1]{\rhdrx{1}{##1}}				% set header for single right column
 \newcommand{\ohdr}[1]{\ohdrx{1}{##1}}
 \newcommand{\br}{\hfill\break}						% force newline within table entry

%% column styles
 \newcolumntype{x}[1]{>{\raggedright\let\\=\tabularnewline}p{##1}}	% word-wrapped ragged-right
 																	% column, width specified by #1
 % \newcolumntype{m}[1]{>{\CodeStyle}l{##1}}              % variable width column, all entries in CodeStyle
 \newcolumntype{m}[1]{l{##1}}							% variable width column, all entries in CodeStyle

  % do not number bullets within tables
  \renewcommand{\labelitemi}{---}
  \renewcommand{\labelitemii}{---}
  \renewcommand{\labelitemiii}{---}
  \renewcommand{\labelitemiv}{---}
}
{
}

% General Usage: TITLE is the title of the table, XREF is the
% cross-reference for the table. LAYOUT is a sequence of column
% type specifiers (e.g. cp{1.0}c), without '|' for the left edge
% or right edge.

% usage: \begin{floattablebase}{TITLE}{XREF}{COLUMNS}{PLACEMENT}
% produces floating table, location determined within limits
% by LaTeX.
\newenvironment{floattablebase}[4]
{
 \begin{TableBase}
 \begin{table}[#4]
 \caption{\label{#2}#1}
 \begin{center}
 \begin{tabular}{|#3|}
}
{
 \bottomline
 \end{tabular}
 \end{center}
 \end{table}
 \end{TableBase}
}

% usage: \begin{floattable}{TITLE}{XREF}{COLUMNS}
% produces floating table, location determined within limits
% by LaTeX.
\newenvironment{floattable}[3]
{
 \begin{floattablebase}{#1}{#2}{#3}{htbp}
}
{
 \end{floattablebase}
}

% a column in a multicolfloattable (internal)
\newenvironment{mcftcol}{%
 \renewcommand{\columnbreak}{%
  \end{mcftcol} &
  \begin{mcftcol}
 }%
 \setlength{\tabcolsep}{0pt}%
 \begin{tabular}[t]{l}
}{
 \end{tabular}
}

% usage: \begin{multicolfloattable}{TITLE}{XREF}{COLUMNS}
% produces floating table, location determined within limits
% by LaTeX.
\newenvironment{multicolfloattable}[3]
{
 \begin{floattable}{#1}{#2}{#3}
 \topline
 \begin{mcftcol}
}
{
 \end{mcftcol} \\
 \end{floattable}
}

% usage: \begin{tokentable}{TITLE}{XREF}{HDR1}{HDR2}
% produces six-column table used for lists of replacement tokens;
% the columns are in pairs -- left-hand column has header HDR1,
% right hand column has header HDR2; pairs of columns are separated
% by vertical lines. Used in the "Alternative tokens" table.
\newenvironment{tokentable}[4]
{
 \begin{floattablebase}{#1}{#2}{cc|cc|cc}{htbp}
 \topline
 \hdstyle{#3}   &   \hdstyle{#4}    &
 \hdstyle{#3}   &   \hdstyle{#4}    &
 \hdstyle{#3}   &   \hdstyle{#4}    \\ \capsep
}
{
 \end{floattablebase}
}

% usage: \begin{libsumtabbase}{TITLE}{XREF}{HDR1}{HDR2}
% produces three-column table with column headers HDR1 and HDR2.
% Used in "Library Categories" table in standard, and used as
% base for other library summary tables.
\newenvironment{libsumtabbase}[4]
{
 \begin{floattable}{#1}{#2}{lll}
 \topline
 \lhdrx{2}{#3}	&	\hdstyle{#4}	\\ \capsep
}
{
 \end{floattable}
}

% usage: \begin{libsumtab}{TITLE}{XREF}
% produces three-column table with column headers "Subclause" and "Header(s)".
% Used in "C++ Headers for Freestanding Implementations" table in standard.
\newenvironment{libsumtab}[2]
{
 \begin{libsumtabbase}{#1}{#2}{Subclause}{Header(s)}
}
{
 \end{libsumtabbase}
}

% usage: \begin{concepttable}{TITLE}{TAG}{LAYOUT}
% produces table at current location
\newenvironment{concepttable}[3]
{
 \begin{TableBase}
 \begin{table}[!htb]
 \caption[#1]{\label{tab:#2}#1}
 \begin{center}
 \begin{tabular}{|#3|}
}
{
 \bottomline
 \end{tabular}
 \end{center}
 \end{table}
 \end{TableBase}
}

% usage: \begin{simpletypetable}{TITLE}{TAG}{LAYOUT}
% produces table at current location
\newenvironment{simpletypetable}[3]
{
 \begin{TableBase}
 \begin{table}[!htb]
 \caption{#1}\label{#2}
 \begin{center}
 \begin{tabular}{|#3|}
}
{
 \bottomline
 \end{tabular}
 \end{center}
 \end{table}
 \end{TableBase}
}

% usage: \begin{LongTable}{TITLE}{XREF}{LAYOUT}
% produces table that handles page breaks sensibly.
\newenvironment{LongTable}[3]
{
 \newcommand{\continuedcaption}{\caption[]{#1 (continued)}}
 \begin{TableBase}
 \begin{longtable}
 {|#3|}\caption{#1}\label{#2}
}
{
 \bottomline
 \end{longtable}
 \end{TableBase}
}

% usage: \begin{libreqtabN}{TITLE}{XREF}
% produces an N-column breakable table. Used in
% most of the library Clauses for requirements tables.
% Example at "Position type requirements" in the standard.

\newenvironment{libreqtab2}[2]
{
 \begin{LongTable}
 {#1}{#2}
 {lx{.55\hsize}}
}
{
 \end{LongTable}
}

\newenvironment{libreqtab2a}[2]
{
 \begin{LongTable}
 {#1}{#2}
 {x{.30\hsize}x{.64\hsize}}
}
{
 \end{LongTable}
}

\newenvironment{libreqtab3}[2]
{
 \begin{LongTable}
 {#1}{#2}
 {x{.28\hsize}x{.18\hsize}x{.43\hsize}}
}
{
 \end{LongTable}
}

\newenvironment{libreqtab3a}[2]
{
 \begin{LongTable}
 {#1}{#2}
 {x{.28\hsize}x{.33\hsize}x{.29\hsize}}
}
{
 \end{LongTable}
}

\newenvironment{libreqtab3b}[2]
{
 \begin{LongTable}
 {#1}{#2}
 {x{.40\hsize}x{.25\hsize}x{.25\hsize}}
}
{
 \end{LongTable}
}

\newenvironment{libreqtab3e}[2]
{
 \begin{LongTable}
 {#1}{#2}
 {x{.38\hsize}x{.27\hsize}x{.25\hsize}}
}
{
 \end{LongTable}
}

\newenvironment{libreqtab3f}[2]
{
 \begin{LongTable}
 {#1}{#2}
 {x{.35\hsize}x{.28\hsize}x{.29\hsize}}
}
{
 \end{LongTable}
}

\newenvironment{libreqtab4a}[2]
{
 \begin{LongTable}
 {#1}{#2}
 {x{.14\hsize}x{.30\hsize}x{.30\hsize}x{.14\hsize}}
}
{
 \end{LongTable}
}

\newenvironment{libreqtab4b}[2]
{
 \begin{LongTable}
 {#1}{#2}
 {x{.13\hsize}x{.15\hsize}x{.29\hsize}x{.27\hsize}}
}
{
 \end{LongTable}
}

\newenvironment{libreqtab4c}[2]
{
 \begin{LongTable}
 {#1}{#2}
 {x{.16\hsize}x{.21\hsize}x{.21\hsize}x{.30\hsize}}
}
{
 \end{LongTable}
}

\newenvironment{libreqtab4d}[2]
{
 \begin{LongTable}
 {#1}{#2}
 {x{.22\hsize}x{.22\hsize}x{.30\hsize}x{.15\hsize}}
}
{
 \end{LongTable}
}

\newenvironment{libreqtab5}[2]
{
 \begin{LongTable}
 {#1}{#2}
 {x{.14\hsize}x{.14\hsize}x{.20\hsize}x{.20\hsize}x{.14\hsize}}
}
{
 \end{LongTable}
}

% usage: \begin{libtab2}{TITLE}{XREF}{LAYOUT}{HDR1}{HDR2}
% produces two-column table with column headers HDR1 and HDR2.
% Used in "seekoff positioning" in the standard.
\newenvironment{libtab2}[5]
{
 \begin{floattable}
 {#1}{#2}{#3}
 \topline
 \lhdr{#4}	&	\rhdr{#5}	\\ \capsep
}
{
 \end{floattable}
}

% usage: \begin{longlibtab2}{TITLE}{XREF}{LAYOUT}{HDR1}{HDR2}
% produces two-column table with column headers HDR1 and HDR2.
\newenvironment{longlibtab2}[5]
{
 \begin{LongTable}{#1}{#2}{#3}
 \\ \topline
 \lhdr{#4}	&	\rhdr{#5}	\\ \capsep
 \endfirsthead
 \continuedcaption\\
 \topline
 \lhdr{#4}	&	\rhdr{#5}	\\ \capsep
 \endhead
}
{
  \end{LongTable}
}

% usage: \begin{LibEffTab}{TITLE}{XREF}{HDR2}{WD2}
% produces a two-column table with left column header "Element"
% and right column header HDR2, right column word-wrapped with
% width specified by WD2.
\newenvironment{LibEffTab}[4]
{
 \begin{libtab2}{#1}{#2}{lp{#4}}{Element}{#3}
}
{
 \end{libtab2}
}

% Same as LibEffTab except that it uses a long table.
\newenvironment{longLibEffTab}[4]
{
 \begin{longlibtab2}{#1}{#2}{lp{#4}}{Element}{#3}
}
{
 \end{longlibtab2}
}

% usage: \begin{libefftab}{TITLE}{XREF}
% produces a two-column effects table with right column
% header "Effect(s) if set", width 4.5 in. Used in "fmtflags effects"
% table in standard.
\newenvironment{libefftab}[2]
{
 \begin{LibEffTab}{#1}{#2}{Effect(s) if set}{4.5in}
}
{
 \end{LibEffTab}
}

% Same as libefftab except that it uses a long table.
\newenvironment{longlibefftab}[2]
{
 \begin{longLibEffTab}{#1}{#2}{Effect(s) if set}{4.5in}
}
{
 \end{longLibEffTab}
}

% usage: \begin{libefftabmean}{TITLE}{XREF}
% produces a two-column effects table with right column
% header "Meaning", width 4.5 in. Used in "seekdir effects"
% table in standard.
\newenvironment{libefftabmean}[2]
{
 \begin{LibEffTab}{#1}{#2}{Meaning}{4.5in}
}
{
 \end{LibEffTab}
}

% usage: \begin{libefftabvalue}{TITLE}{XREF}
% produces a two-column effects table with right column
% header "Value", width 3 in. Used in "basic_ios::init() effects"
% table in standard.
\newenvironment{libefftabvalue}[2]
{
 \begin{LibEffTab}{#1}{#2}{Value}{3in}
}
{
 \end{LibEffTab}
}

% usage: \begin{libefftabvaluenarrow}{TITLE}{XREF}
% produces a two-column effects table with right column
% header "Value", width 1.5 in. Used in basic_string_view effects
% tables in standard.
\newenvironment{libefftabvaluenarrow}[2]
{
 \begin{LibEffTab}{#1}{#2}{Value}{1.5in}
}
{
 \end{LibEffTab}
}

% Same as libefftabvalue except that it uses a long table and a
% slightly wider column.
\newenvironment{longlibefftabvalue}[2]
{
 \begin{longLibEffTab}{#1}{#2}{Value}{3.5in}
}
{
 \end{longLibEffTab}
}

% usage: \begin{liberrtab}{TITLE}{XREF} produces a two-column table
% with left column header ``Value'' and right header "Error
% condition", width 4.5 in. Used in regex Clause in the TR.

\newenvironment{liberrtab}[2]
{
 \begin{libtab2}{#1}{#2}{lp{4.5in}}{Value}{Error condition}
}
{
 \end{libtab2}
}

% Like liberrtab except that it uses a long table.
\newenvironment{longliberrtab}[2]
{
 \begin{longlibtab2}{#1}{#2}{lp{4.5in}}{Value}{Error condition}
}
{
 \end{longlibtab2}
}

% usage: \begin{lib2dtab2base}{TITLE}{XREF}{HDR1}{HDR2}{WID1}{WID2}{WID3}
% produces a table with one heading column followed by 2 data columns.
% used for 2D requirements tables, such as optional::operator= effects
% tables.
\newenvironment{lib2dtab2base}[7]
{
 %% no lines in the top-left cell, and leave a gap around the headers
 %% FIXME: I tried to use hhline here, but it doesn't appear to support
 %% the join between the leftmost top header and the topmost left header,
 %% so we fake it with an empty row and column.
 \newcommand{\topline}{\cline{3-4}}
 \newcommand{\rowsep}{\cline{1-1}\cline{3-4}}
 \newcommand{\capsep}{
  \topline
  \multicolumn{4}{c}{}\\[-0.8\normalbaselineskip]
  \rowsep
 }
 \newcommand{\bottomline}{\rowsep}
 \newcommand{\hdstyle}[1]{\textbf{##1}}
 \newcommand{\rowhdr}[1]{\hdstyle{##1}&}
 \newcommand{\colhdr}[1]{\multicolumn{1}{|>{\centering}m{#6}|}{\hdstyle{##1}}}
 %% FIXME: figure out a way to reuse floattable here
 \begin{table}[htbp]
 \caption{\label{#2}#1}
 \begin{center}
 \begin{tabular}{|>{\centering}m{#5}|@{}p{0.2\normalbaselineskip}@{}|m{#6}|m{#7}|}
 %% table header
 \topline
 \multicolumn{1}{c}{}&&\colhdr{#3}&\colhdr{#4}\\
 \capsep
}
{
 \bottomline
 \end{tabular}
 \end{center}
 \end{table}
}

\newenvironment{lib2dtab2}[4]{
 \begin{lib2dtab2base}{#1}{#2}{#3}{#4}{1.2in}{1.8in}{1.8in}
}{
 \end{lib2dtab2base}
}

% Define section from the C++ standard that can be indexed
% using its dotted identifer. That is:
%
%  \cxxsec{basic.def.odr}{3.2}
%
% This is used to make references to sections of the C++ Standard
% that are not labeled within this document.
\newcommand{\cxxsec}[2]{%
  \expandafter\def\csname #1 \endcsname{#2}%
}

% Generate a reference to the section with the given id. This
% expands to the full chapter/section/subsection number declared
% by \cxxsec. For example:
%
%  \cxxref{basic.def.odr}
%
% Expands to the string 3.2.
\newcommand{\stdcxxref}[1]{%
  \csname #1 \endcsname%
}

\newcommand{\cxxref}[1]{%
  \stdcxxref{#1}%
}

\newcommand{\tsref}[1]{%
  ISO/IEC TS 21425:2017 \S\stdcxxref{#1}%
}

% From the Ranges TS
\cxxsec{func.identity}{8.3.3}
\cxxsec{meta.trans.other}{8.4.3}

\cxxsec{intro.compliance}{4.1}
\cxxsec{intro.execution}{6.8.1}
\cxxsec{intro.multithread}{6.8.2}

\cxxsec{lex.key}{5.11}

\cxxsec{basic.def.odr}{6.2}
\cxxsec{basic.scope.namespace}{6.3.6}
\cxxsec{basic.lookup}{6.4}
\cxxsec{basic.lookup.unqual}{6.4.1}
\cxxsec{basic.lookup.argdep}{6.4.2}
\cxxsec{basic.lookup.classref}{6.4.5}
\cxxsec{basic.types}{6.7}
\cxxsec{basic.fundamental}{6.7.1}
\cxxsec{basic.compound}{6.7.2}
\cxxsec{basic.type.qualifier}{6.7.3}

\cxxsec{conv}{7}
\cxxsec{conv.array}{7.2}
\cxxsec{conv.integral}{7.8}

\cxxsec{expr}{8}
\cxxsec{expr.prim}{8.4}
\cxxsec{expr.prim.lambda}{8.4.5}
\cxxsec{expr.prim.fold}{8.4.6}
\cxxsec{expr.prim.req}{8.4.7}
\cxxsec{expr.typeid}{8.5.1.8}
\cxxsec{expr.unary}{8.5.2}
\cxxsec{expr.unary.op}{8.5.2.1}
\cxxsec{expr.pre.incr}{8.5.2.2}
\cxxsec{expr.sizeof}{8.5.2.3}
\cxxsec{expr.new}{8.5.2.4}
\cxxsec{expr.delete}{8.5.2.5}
\cxxsec{expr.alignof}{8.5.2.6}
\cxxsec{expr.unary.noexcept}{8.5.2.7}
\cxxsec{expr.cast}{8.5.3}
\cxxsec{expr.mptr.oper}{8.5.4}
\cxxsec{expr.mul}{8.5.5}
\cxxsec{expr.add}{8.5.6}
\cxxsec{expr.shift}{8.5.7}
\cxxsec{expr.rel}{8.5.9}
\cxxsec{expr.eq}{8.5.10}
\cxxsec{expr.bit.and}{8.5.11}
\cxxsec{expr.xor}{8.5.12}
\cxxsec{expr.or}{8.5.13}
\cxxsec{expr.log.and}{8.5.14}
\cxxsec{expr.log.or}{8.5.15}
\cxxsec{expr.cond}{8.5.16}
\cxxsec{expr.throw}{8.5.17}
\cxxsec{expr.ass}{8.5.18}
\cxxsec{expr.comma}{8.5.19}
\cxxsec{expr.const}{8.6}

\cxxsec{stmt.block}{9.3}
\cxxsec{stmt.select}{9.4}
\cxxsec{stmt.iter}{9.5}
\cxxsec{stmt.for}{9.5.3}

\cxxsec{dcl.dcl}{10}
\cxxsec{dcl.spec}{10.1}
\cxxsec{dcl.fct.spec}{10.1.2}
\cxxsec{dcl.constexpr}{10.1.5}
\cxxsec{dcl.type}{10.1.7}
\cxxsec{dcl.type.cv}{10.1.7.1}
\cxxsec{dcl.type.simple}{10.1.7.2}
\cxxsec{dcl.type.elab}{10.1.6.3}
\cxxsec{dcl.spec.auto}{10.1.6.4}
\cxxsec{namespace.alias}{10.3.2}
\cxxsec{namespace.udecl}{10.3.3}
\cxxsec{dcl.attr.deprecated}{10.6.4}

\cxxsec{dcl.decl}{11}
\cxxsec{dcl.name}{11.1}
\cxxsec{dcl.ambig.res}{11.2}
\cxxsec{dcl.meaning}{11.3}
\cxxsec{dcl.fct}{11.3.5}
\cxxsec{dcl.fct.default}{11.3.6}
\cxxsec{dcl.fct.def}{11.4}
\cxxsec{dcl.fct.def.general}{11.4.1}
\cxxsec{dcl.init}{11.6}
\cxxsec{dcl.init.list}{11.6.4}

\cxxsec{class}{12}
\cxxsec{class.name}{12.1}
\cxxsec{class.mem}{12.2}
\cxxsec{class.static.data}{12.2.3.2}

\cxxsec{class.derived}{13}
\cxxsec{class.mi}{13.1}
\cxxsec{class.member.lookup}{13.2}
\cxxsec{class.virtual}{13.3}
\cxxsec{class.abstract}{13.4}

\cxxsec{class.ctor}{15.1}
\cxxsec{class.conv}{15.3}
\cxxsec{class.dtor}{15.4}

\cxxsec{over}{16}
\cxxsec{over.load}{16.1}
\cxxsec{over.dcl}{16.2}
\cxxsec{over.match}{16.3}
\cxxsec{over.match.funcs}{16.3.1}
\cxxsec{over.match.class.deduct}{16.3.1.8}
\cxxsec{over.match.viable}{16.3.2}
\cxxsec{over.match.best}{16.3.3}
\cxxsec{over.over}{16.4}
\cxxsec{over.oper}{16.5}
\cxxsec{over.call}{16.5.4}
\cxxsec{over.built}{16.6}

\cxxsec{temp}{17}
\cxxsec{temp.param}{17.1}
\cxxsec{temp.names}{17.2}
\cxxsec{temp.arg}{17.3}
\cxxsec{temp.arg.type}{17.3.1}
\cxxsec{temp.arg.nontype}{17.3.2}
\cxxsec{temp.arg.template}{17.3.3}
\cxxsec{temp.constr}{17.4}
\cxxsec{temp.type}{17.5}
\cxxsec{temp.dcls}{17.6}
\cxxsec{temp.class}{17.6.1}
\cxxsec{temp.mem.func}{17.6.1.1}
\cxxsec{temp.mem.class}{17.6.1.2}
\cxxsec{temp.static}{17.6.1.3}
\cxxsec{temp.mem.enum}{17.6.1.4}
\cxxsec{temp.mem}{17.6.2}
\cxxsec{temp.variadic}{17.6.3}
\cxxsec{temp.friend}{17.6.4}
\cxxsec{temp.class.spec.match}{17.6.5.1}
\cxxsec{temp.class.order}{17.6.5.2}
\cxxsec{temp.class.spec.mfunc}{17.6.5.3}
\cxxsec{temp.fct}{17.6.6}
\cxxsec{temp.over.link}{17.6.6.1}
\cxxsec{temp.func.order}{17.6.6.2}
\cxxsec{temp.res}{17.7}
\cxxsec{temp.local}{17.7.1}
\cxxsec{temp.dep}{17.7.2}
\cxxsec{temp.dep.type}{17.7.2.1}
\cxxsec{temp.dep.expr}{17.7.2.2}
\cxxsec{temp.dep.constexpr}{17.7.2.3}
\cxxsec{temp.dep.temp}{17.7.2.4}
\cxxsec{temp.nondep}{17.7.3}
\cxxsec{temp.dep.res}{17.7.4}
\cxxsec{temp.point}{17.7.4.1}
\cxxsec{temp.inject}{17.7.5}
\cxxsec{temp.spec}{17.8}
\cxxsec{temp.inst}{17.8.1}
\cxxsec{temp.explicit}{17.8.2}
\cxxsec{temp.expl.spec}{17.8.3}
\cxxsec{temp.deduct}{17.9.2}
\cxxsec{temp.deduct.call}{17.9.2.1}
\cxxsec{temp.deduct.funcaddr}{17.9.2.2}
\cxxsec{temp.deduct.conv}{17.9.2.3}
\cxxsec{temp.deduct.partial}{17.9.2.4}
\cxxsec{temp.deduct.type}{17.9.2.5}
\cxxsec{temp.deduct.decl}{17.9.2.6}

\cxxsec{except}{18}
\cxxsec{except.throw}{18.1}
\cxxsec{except.ctor}{18.2}
\cxxsec{except.handle}{18.3}
\cxxsec{except.spec}{18.4}
\cxxsec{except.special}{18.5}
\cxxsec{except.terminate}{18.5.1}
\cxxsec{except.uncaught}{18.5.2}

% \cxxsec{library}{20}
% \cxxsec{type.descriptions}{20.4.2.1}
% \cxxsec{operators}{20.4.2.3}
% \cxxsec{requirements}{20.5}
% \cxxsec{utility.requirements}{20.5.3}
% \cxxsec{utility.arg.requirements}{20.5.3.1}
% \cxxsec{conforming}{20.5.5}
% \cxxsec{algorithm.stable}{20.5.5.7}
% \cxxsec{res.on.exception.handling}{20.5.5.12}

% \cxxsec{support}{21}
% \cxxsec{support.initlist}{21.9}

\cxxsec{library}{20.1}
\cxxsec{type.descriptions}{20.1.4.2.1}
\cxxsec{customization.point.object}{20.1.4.2.1.6}
\cxxsec{operators}{20.1.4.2.3}
\cxxsec{requirements}{20.1.5}
\cxxsec{utility.requirements}{20.1.5.3}
\cxxsec{utility.arg.requirements}{20.1.5.3.1}
\cxxsec{conforming}{20.1.5.5}
\cxxsec{algorithm.stable}{20.1.5.5.7}
\cxxsec{res.on.exception.handling}{20.1.5.5.12}

\cxxsec{support}{20.2}
\cxxsec{support.initlist}{20.2.9}

\cxxsec{tab:equality.comparable.requirements}{20}
\cxxsec{tab:less.than.comparable.requirements}{21}
\cxxsec{tab:move.constructible.requirements}{23}
\cxxsec{tab:copy.constructible.requirements}{24}
\cxxsec{tab:move.assignable.requirements}{25}

% \cxxsec{utility.swap}{23.2.2}
% \cxxsec{pairs}{23.4}
% \cxxsec{tuple}{23.5}
% \cxxsec{tuple.helper}{23.5.3.6}
% \cxxsec{tuple.elem}{23.5.3.7}
% \cxxsec{func.def}{23.14.2}
% \cxxsec{func.require}{23.14.3}
% \cxxsec{refwrap}{23.14.5}
% \cxxsec{meta}{23.15}
% \cxxsec{meta.unary}{23.15.4}
% \cxxsec{meta.unary.prop}{23.15.4.3}

% \cxxsec{containers}{26}
% \cxxsec{associative}{26.4}
% \cxxsec{set}{26.4.6}
% \cxxsec{multiset}{26.4.7}
% \cxxsec{unord.set}{26.5.6}
% \cxxsec{unord.multiset}{26.5.7}

% \cxxsec{iterator.traits}{27.4.1}

\cxxsec{concepts.lib.general.equality}{20.3.1.1}
\cxxsec{concepts.lib.corelang.swappable}{20.3.3.11}
\cxxsec{concepts.lib.compare.equalitycomparable}{20.3.4.3}
\cxxsec{concepts.lib.compare.stricttotallyordered}{20.3.4.4}

\cxxsec{utility.swap}{20.5.2.2}
\cxxsec{pairs}{20.5.4}
\cxxsec{tuple}{20.5.5}
\cxxsec{tuple.helper}{20.5.5.3.6}
\cxxsec{tuple.elem}{20.5.5.3.7}
\cxxsec{func.def}{20.5.14.2}
\cxxsec{func.require}{20.5.14.3}
\cxxsec{refwrap}{20.5.14.5}
\cxxsec{meta}{20.5.15}
\cxxsec{meta.unary}{20.5.15.4}
\cxxsec{meta.unary.prop}{20.5.15.4.3}

\cxxsec{containers}{20.8}
\cxxsec{associative}{20.8.4}
\cxxsec{set}{20.8.4.6}
\cxxsec{multiset}{20.8.4.7}
\cxxsec{unord.set}{20.8.5.6}
\cxxsec{unord.multiset}{20.8.5.7}

\cxxsec{iterator.traits}{20.9.4.1}

\cxxsec{tab:iterator.forward.requirements}{89}
\cxxsec{tab:iterator.bidirectional.requirements}{90}
\cxxsec{tab:iterator.random.access.requirements}{91}

% \cxxsec{numerics}{29}
% \cxxsec{rand}{29.6}
% \cxxsec{rand.req}{29.6.1}
% \cxxsec{rand.req.urng}{29.6.1.3}

% \cxxsec{input.output}{30}
% \cxxsec{stream.buffers}{30.6}
% \cxxsec{iostream.format}{30.7}

\cxxsec{numerics}{20.11}
\cxxsec{rand}{20.11.6}
\cxxsec{rand.req}{20.11.6.1}
\cxxsec{rand.req.urng}{20.11.6.1.3}

\cxxsec{input.output}{20.12}
\cxxsec{stream.buffers}{20.12.6}
\cxxsec{iostream.format}{20.12.7}

\cxxsec{tab:iterator.forward.requirements}{76}


%%--------------------------------------------------
%% fix interaction between hyperref and other
%% commands
\pdfstringdefDisableCommands{\def\smaller#1{#1}}
\pdfstringdefDisableCommands{\def\textbf#1{#1}}
\pdfstringdefDisableCommands{\def\raisebox#1{}}
\pdfstringdefDisableCommands{\def\hspace#1{}}

%%--------------------------------------------------
%% add special hyphenation rules
\hyphenation{tem-plate ex-am-ple in-put-it-er-a-tor name-space name-spaces}

\begin{document}
\chapterstyle{cppstd}
\pagestyle{cpppage}

%%--------------------------------------------------
%% configuration
%!TEX root = D0896.tex
%%--------------------------------------------------
%% Version numbers
\newcommand{\docno}{P1223R2}
\newcommand{\cppver}{202000L}

%% Release date
\newcommand{\reldate}{\today}

\newcommand{\firstlibchapter}{std2.utilities}
\newcommand{\lastlibchapter}{std2.numerics}


%%--------------------------------------------------
%% front matter
\frontmatter
%!TEX root = D0896.tex
%!TEX root = D0896.tex
%%--------------------------------------------------
%% Title page for the C++ Standard

\thispagestyle{empty}
\begingroup
\def\hd{\begin{tabular}{lll}
          \textbf{Document Number:} & {\larger\docno}               \\
          \textbf{Date:}            & \reldate                      \\
          \textbf{Reply to:}        & Zach Laine                    \\
                                    & whatwasthataddress@gmail.com  \\
          \textbf{Audience:}        & LEWG, LWG                     \\
          \end{tabular}
}
\newlength{\hdwidth}
\settowidth{\hdwidth}{\hd}
\hfill\begin{minipage}{\hdwidth}\hd\end{minipage}
\endgroup

\title{\textbf{\Huge \tcode{find_last}}}
\date{}
{\let\newpage\relax\maketitle}

% \input{cover-reg}

Wording in this paper applies to N4820.

%%--------------------------------------------------
%% The table of contents, list of tables, and list of figures
\markboth{\contentsname}{}

%%--------------------------------------------------
%% Make a bit more room for our long page numbers.
\makeatletter
\renewcommand\@pnumwidth{2.5em}
\makeatother

\tableofcontents
\setcounter{tocdepth}{5}
% \newpage
% \listoftables
% \newpage
% \listoffigures

%\input{preface}


%%--------------------------------------------------
%% main body of the document
\mainmatter
\setglobalstyles

\section{Revisions}

\subsection{Changes from R1}

\begin{itemize}
  \item Change \tcode{find_backward()} to \tcode{find_last()}.
  \item Wording.
\end{itemize}

\subsection{Changes from R0}

\begin{itemize}
  \item Base synopsis on The One Ranges Proposal (P0896R4).
  \item Drop \tcode{std}-namespace overloads.
  \item Drop \tcode{find_not()} and \tcode{find_not_backward()}.
\end{itemize}


\section{Motivation and Scope}

Consider how finding the last element that is equal to `x` in a range is
typically done (for all the examples below, we assume a valid range of
elements \tcode{[first, last)}, and an iterator \tcode{it} within that range):

\begin{itemdecl}
    while (it-- != first) {
        if (*it == x) {
            // Use it here...
        }
    }
\end{itemdecl}

Raw loops are icky though.  Perhaps we should do a bit of extra work to allow
the use of \tcode{find()}:

\begin{itemdecl}
    auto rfirst = std::make_reverse_iterator(it);
    auto rlast = std::make_reverse_iterator(first);
    auto it = std::find(rfirst, rlast, x);
    // Use it here...
\end{itemdecl}

That seems nicer in that there is no raw loop, but it requires an unpleasant
amount of typing (and an associated lack of clarity).

Consider this instead:

\begin{itemdecl}
    auto it = std::find_last(first, it, x);
    // Use it here...
\end{itemdecl}

That's better!  It's a lot less verbose.

Let's consider for a moment the lack of clarity of the
\tcode{make_reverse_iterator()} code.  In a typical use of \tcode{find()}, I
search forward from the element I start from, including the element itself:

\begin{itemdecl}
    auto it = std::find(it, last, x); // Includes examination of *it.
\end{itemdecl}

However, using finding in reverse in the middle of a range leaves out the
element pointed to by the current iterator:

\begin{itemdecl}
    auto it = std::find( // Skips *it entirely.
        std::make_reverse_iterator(first),
        std::make_reverse_iterator(it),
        x);
\end{itemdecl}

That leads to code like this:

\begin{itemdecl}
    auto it = std::find( // Includes *it again!
        std::make_reverse_iterator(first),
        std::make_reverse_iterator(std::next(it)),
        x);
\end{itemdecl}

Though this looks like an off-by-one error, is is correct.  Moreover, even
though the use of \tcode{next()} is correct, it gets lost in noise of the rest
of the code, since it is so verbose.  Use \tcode{find_last()} makes things
clearer:

\begin{itemdecl}
    // Search, but don't include *it.
    auto it_1 = std::find_last(first, it, x);

    // Search, and include *it.
    auto it_2 = std::find_last(first, std::next(it), x);
\end{itemdecl}

The use of \tcode{next()} may at first appear like a mistake, until the reader
takes a moment to think things through.  In the \tcode{reverse_iterator}
version, this correctness is a lot harder to readily grasp.

\section{Feature-Test Macro}

In addition to the wording that follows, add a new feature-test macro
\tcode{__cpp_lib_find_last}, with value \tcode{202012}.  This macro should be
defined in \tcode{<algorithm>}.

\newcommand{\indexhdr}[1]{}
\newcommand{\indexlibrarymember}[2]{
}
%!TEX root = std.tex
\setcounter{chapter}{24}
\setcounter{section}{5}
\setcounter{subsection}{1}
\setcounter{subsubsection}{1}

\rSec0[algorithms]{Algorithms library}

\setcounter{section}{3}

\rSec1[algorithm.syn]{Header \tcode{<algorithm>} synopsis}
\indexhdr{algorithm}%

\begin{codeblock}
#include <initializer_list>

namespace std {
  // \ref{alg.nonmodifying}, non-modifying sequence operations

  // \ref{alg.find}, find
  template<class InputIterator, class T>
    constexpr InputIterator find(InputIterator first, InputIterator last,
                                 const T& value);
  template<class ExecutionPolicy, class ForwardIterator, class T>
    ForwardIterator find(ExecutionPolicy&& exec, // see \ref{algorithms.parallel.overloads}
                         ForwardIterator first, ForwardIterator last,
                         const T& value);
  template<class InputIterator, class Predicate>
    constexpr InputIterator find_if(InputIterator first, InputIterator last,
                                    Predicate pred);
  template<class ExecutionPolicy, class ForwardIterator, class Predicate>
    ForwardIterator find_if(ExecutionPolicy&& exec, // see \ref{algorithms.parallel.overloads}
                            ForwardIterator first, ForwardIterator last,
                            Predicate pred);
  template<class InputIterator, class Predicate>
    constexpr InputIterator find_if_not(InputIterator first, InputIterator last,
                                        Predicate pred);
  template<class ExecutionPolicy, class ForwardIterator, class Predicate>
    ForwardIterator find_if_not(ExecutionPolicy&& exec, // see \ref{algorithms.parallel.overloads}
                                ForwardIterator first, ForwardIterator last,
                                Predicate pred);

  namespace ranges {
    template<input_iterator I, sentinel<I> S, class T, class Proj = identity>
      requires indirect_binary_predicate<ranges::equal_to, projected<I, Proj>, const T*>
      constexpr I find(I first, S last, const T& value, Proj proj = {});
    template<input_range R, class T, class Proj = identity>
      requires indirect_binary_predicate<ranges::equal_to, projected<iterator_t<R>, Proj>, const T*>
      constexpr borrowed_iterator_t<R>
        find(R&& r, const T& value, Proj proj = {});
    template<input_iterator I, sentinel<I> S, class Proj = identity,
             indirect_unary_predicate<projected<I, Proj>> Pred>
      constexpr I find_if(I first, S last, Pred pred, Proj proj = {});
    template<input_range R, class Proj = identity,
             indirect_unary_predicate<projected<iterator_t<R>, Proj>> Pred>
      constexpr borrowed_iterator_t<R>
        find_if(R&& r, Pred pred, Proj proj = {});
    template<input_iterator I, sentinel<I> S, class Proj = identity,
             indirect_unary_predicate<projected<I, Proj>> Pred>
      constexpr I find_if_not(I first, S last, Pred pred, Proj proj = {});
    template<input_range R, class Proj = identity,
             indirect_unary_predicate<projected<iterator_t<R>, Proj>> Pred>
      constexpr borrowed_iterator_t<R>
        find_if_not(R&& r, Pred pred, Proj proj = {});
  }
\end{codeblock}
\begin{addedblock}
\begin{codeblock}

  // \ref{alg.find.last}, find last
  namespace ranges {
    template<forward_iterator I, sentinel<I> S, class T, class Proj = identity>
      requires indirect_binary_predicate<ranges::equal_to, projected<I, Proj>, const T*>
      constexpr I find_last(I first, S last, const T& value, Proj proj = {});
    template<forward_range R, class T, class Proj = identity>
      requires indirect_binary_predicate<ranges::equal_to, projected<iterator_t<R>, Proj>, const T*>
      constexpr borrowed_iterator_t<R>
        find_last(R&& r, const T& value, Proj proj = {});
    template<forward_iterator I, sentinel<I> S, class Proj = identity,
             indirect_unary_predicate<projected<I, Proj>> Pred>
      constexpr I find_last_if(I first, S last, Pred pred, Proj proj = {});
    template<forward_range R, class Proj = identity,
             indirect_unary_predicate<projected<iterator_t<R>, Proj>> Pred>
      constexpr borrowed_iterator_t<R>
        find_last_if(R&& r, Pred pred, Proj proj = {});
    template<forward_iterator I, sentinel<I> S, class Proj = identity,
             indirect_unary_predicate<projected<I, Proj>> Pred>
      constexpr I find_last_if_not(I first, S last, Pred pred, Proj proj = {});
    template<forward_range R, class Proj = identity,
             indirect_unary_predicate<projected<iterator_t<R>, Proj>> Pred>
      constexpr borrowed_iterator_t<R>
        find_last_if_not(R&& r, Pred pred, Proj proj = {});
  }
\end{codeblock}
\end{addedblock}
\begin{codeblock}
}
\end{codeblock}

\rSec1[alg.nonmodifying]{Non-modifying sequence operations}

\setcounter{subsection}{4}
\rSec2[alg.find]{Find}

\indexlibrary{\idxcode{find}}%
\indexlibrary{\idxcode{find_if}}%
\indexlibrary{\idxcode{find_if_not}}%
\begin{itemdecl}
template<class InputIterator, class T>
  constexpr InputIterator find(InputIterator first, InputIterator last,
                               const T& value);
template<class ExecutionPolicy, class ForwardIterator, class T>
  ForwardIterator find(ExecutionPolicy&& exec, ForwardIterator first, ForwardIterator last,
                       const T& value);

template<class InputIterator, class Predicate>
  constexpr InputIterator find_if(InputIterator first, InputIterator last,
                                  Predicate pred);
template<class ExecutionPolicy, class ForwardIterator, class Predicate>
  ForwardIterator find_if(ExecutionPolicy&& exec, ForwardIterator first, ForwardIterator last,
                          Predicate pred);

template<class InputIterator, class Predicate>
  constexpr InputIterator find_if_not(InputIterator first, InputIterator last,
                                      Predicate pred);
template<class ExecutionPolicy, class ForwardIterator, class Predicate>
  ForwardIterator find_if_not(ExecutionPolicy&& exec,
                              ForwardIterator first, ForwardIterator last,
                              Predicate pred);

template<InputIterator I, sentinel<I> S, class T, class Proj = identity>
 requires indirect_binary_predicate<ranges::equal_to, projected<I, Proj>, const T*>
 constexpr I ranges::find(I first, S last, const T& value, Proj proj = {});
template<input_range R, class T, class Proj = identity>
 requires indirect_binary_predicate<ranges::equal_to, projected<iterator_t<R>, Proj>, const T*>
 constexpr borrowed_iterator_t<R>
   ranges::find(R&& r, const T& value, Proj proj = {});

template<InputIterator I, sentinel<I> S, class Proj = identity,
         indirect_unary_predicate<projected<I, Proj>> Pred>
  constexpr I ranges::find_if(I first, S last, Pred pred, Proj proj = {});
template<input_range R, class Proj = identity,
         indirect_unary_predicate<projected<iterator_t<R>, Proj>> Pred>
 constexpr borrowed_iterator_t<R>
   ranges::find_if(R&& r, Pred pred, Proj proj = {});

template<InputIterator I, sentinel<I> S, class Proj = identity,
         indirect_unary_predicate<projected<I, Proj>> Pred>
 constexpr I ranges::find_if_not(I first, S last, Pred pred, Proj proj = {});
template<input_range R, class Proj = identity,
         indirect_unary_predicate<projected<iterator_t<R>, Proj>> Pred>
 constexpr borrowed_iterator_t<R>
   ranges::find_if_not(R&& r, Pred pred, Proj proj = {});
\end{itemdecl}

\begin{itemdescr}
\pnum
Let $E$ be:
\begin{itemize}
\item \tcode{*i == value} for \tcode{find},
\item \tcode{pred(*i) != false} for \tcode{find_if},
\item \tcode{pred(*i) == false} for \tcode{find_if_not},
\item \tcode{bool(invoke(proj, *i) == value)} for \tcode{ranges::find};
\item \tcode{bool(invoke(pred, invoke(proj, *i)))} for \tcode{ranges::find_if};
\item \tcode{bool(!invoke(pred, invoke(proj, *i)))} for \tcode{ranges::find_if_not}.
\end{itemize}

\pnum
\returns
The first iterator \tcode{i} in the range \range{first}{last}
for which $E$ is \tcode{true}.
Returns \tcode{last} if no such iterator is found.

\pnum
\complexity
At most \tcode{last - first} applications
of the corresponding predicate and projection.
\end{itemdescr}
\begin{addedblock}

\rSec2[alg.find.last]{Find last}

\indexlibrary{\idxcode{find_last}}%
\indexlibrary{\idxcode{find_last_if}}%
\indexlibrary{\idxcode{find_last_if_not}}%
\begin{itemdecl}
template<forward_iterator I, sentinel<I> S, class T, class Proj = identity>
 requires indirect_binary_predicate<ranges::equal_to, projected<I, Proj>, const T*>
 constexpr I ranges::find_last(I first, S last, const T& value, Proj proj = {});
template<forward_range R, class T, class Proj = identity>
 requires indirect_binary_predicate<ranges::equal_to, projected<iterator_t<R>, Proj>, const T*>
 constexpr borrowed_iterator_t<R>
   ranges::find_last(R&& r, const T& value, Proj proj = {});
template<forward_iterator I, sentinel<I> S, class Proj = identity,
         indirect_unary_predicate<projected<I, Proj>> Pred>
 constexpr I ranges::find_last_if(I first, S last, Pred pred, Proj proj = {});
template<forward_range R, class Proj = identity,
         indirect_unary_predicate<projected<iterator_t<R>, Proj>> Pred>
 constexpr borrowed_iterator_t<R>
   ranges::find_last_if(R&& r, Pred pred, Proj proj = {});
template<forward_iterator I, sentinel<I> S, class Proj = identity,
         indirect_unary_predicate<projected<I, Proj>> Pred>
 constexpr I ranges::find_last_if_not(I first, S last, Pred pred, Proj proj = {});
template<forward_range R, class Proj = identity,
         indirect_unary_predicate<projected<iterator_t<R>, Proj>> Pred>
 constexpr borrowed_iterator_t<R>
   ranges::find_last_if_not(R&& r, Pred pred, Proj proj = {});
\end{itemdecl}

\begin{itemdescr}
\pnum
Let $E$ be:
\begin{itemize}
\item \tcode{bool(invoke(proj, *i) == value)} for \tcode{ranges::find_last};
\item \tcode{bool(invoke(pred, invoke(proj, *i)))} for \tcode{ranges::find_last_if};
\item \tcode{bool(!invoke(pred, invoke(proj, *i)))} for \tcode{ranges::find_last_if_not}.
\end{itemize}

\pnum
\returns
The last iterator \tcode{i} in the range \range{first}{last}
for which $E$ is \tcode{true}.
Returns \tcode{last} if no such iterator is found.

\pnum
\complexity
At most \tcode{last - first} applications
of the corresponding predicate and projection.
\end{itemdescr}
\end{addedblock}


%%--------------------------------------------------
%!TEX root = stl2-ts.tex
\section{Acknowledgements}

Thanks to Alisdair Meredith and Marshall Clow for encouraging this submission.


%%--------------------------------------------------
%% End of document
\end{document}
