\section{Motivation and Scope}

Consider how finding the last element that is equal to `x` in a range is
typically done (for all the examples below, we assume a valid range of
elements \code{[first, last)}, and an iterator \code{it} within that range):

\lstinputlisting[language=C++, firstline=3, lastline=7]{snippets.cpp}

Raw loops are icky though.  Perhaps we should do a bit of extra work to allow
the use of \code{find()}:

\lstinputlisting[language=C++, firstline=12, lastline=15]{snippets.cpp}

That seems nicer in that there is no raw loop, but it requires an unpleasant
amount of typing (and an associated lack of clarity).

Consider this instead:

\lstinputlisting[language=C++, firstline=41, lastline=42]{snippets.cpp}

That's better!  It's a lot less verbose.

Let's consider for a moment the lack of clarity of the
\code{make_reverse_iterator()} code.  In a typical use of \code{find()}, I
search forward from the element I start from, including the element itself:

\lstinputlisting[language=C++, firstline=20, lastline=20]{snippets.cpp}

However, using finding in reverse in the middle of a range leaves out the
element pointed to by the current iterator:

\lstinputlisting[language=C++, firstline=25, lastline=28]{snippets.cpp}

That leads to code like this:

\lstinputlisting[language=C++, firstline=33, lastline=36]{snippets.cpp}

Though this looks like an off-by-one error. is is correct.  Moreover, even
though the use of \code{next()} is correct, it gets lost in noise of the rest
of the code, since it is so verbose.  Use \code{find_backward()} makes things
clearer:

\lstinputlisting[language=C++, firstline=47, lastline=51]{snippets.cpp}

The use of \code{next()} may at first appear like a mistake, until the reader
takes a moment to think things through.  In the \code{reverse_iterator}
version, this correctness is a lot harder to readily grasp.
